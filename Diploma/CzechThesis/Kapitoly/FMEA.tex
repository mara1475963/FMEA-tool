\chapter{FMEA}
\label{sec:FMEA}
FMEA je zkratka z anglického výrazu Failure Mode and Effect Analysis, česky lze toto označení přeložit jako Analýza možných způsobů a důsledků závad nebo jako Analýza možného výskytu a vlivu vad \cite{preklad}. Jedná se tedy o analytickou metodu, která ma za cíl odhalení možných vad ve výrobním procesu nebo návrhu produktu, nalezení příčin výskytu těchto vad, ohodnocení závažnosti daného rizika a snahu o jeho zmírnění nebo odstranění. Na vypracování této analýzy se většinou podílí tým odborníků z různých oblastí daného odvětví, kteří využívají svých znalostí a zkušeností ze svých profesí k odbornému posouzení problémů v různých fázích analýzy. Obvykle tento tým tvoří vedoucí výroby, zástupce z oddělení pro kontrolu kvality, konstruktér, technolog popřípadě další odborníci. Ideální počet členu by se měl pohybovat okolo 4-6 jedinců. \cite{fmeaTeam} Mezi relevantní účastníky také může patřit zákazník, kterému mohou být poskytovány výstupy z jednotlivých verzí analýzy.  V tomto případě může být FMEA součástí smluvní dohody mezi výrobcem a zákazníkem jako záruka kvality například v rámci PPAP. 

\section{Historie FMEA}
\label{sec:historie}
 Počátky FMEA sahají do 40. let minulého století, kdy byla analýza poprvé použita americkou armádou pro redukci potenciálních závad při výrobě munice. Metoda se ukázala jako vysoce efektivní a kolem roku 1960 ji začlenila do svých přípravných technik i společnost NASA. FMEA se ukázala jako podstatná součást mise Apollo. Od 70.let 20. století již následoval automobilový průmysl, který tvoří jedno z hlavních odvětví, kde je tato metoda využívána. Prvotním uživatelem byla automobilka Ford, která přijala metodu jako reakci na špatný bezpečnostní stav jejich modelu Ford Pinto. Jejich příkladu pak následovali další američtí i evropští výrobci. Tyto události vedly ke vzniku asociací AIAG \cite{aiag} a VDA \cite{vda}, které definují standardy pro zvýšení kvality pomocí nástrojů jako je FMEA, SPC nebo MSA. \cite{historie}

\section{Aplikace FMEA}
 Odpověď na otázku, kdy se FMEA má provádět, se může lišit případ od případu. Nicméně prvotním záměrem je začít provádět metodu, co nejdříve, co jsou k dispozici potřebné informace. Konkrétně se může jednat o případy, kdy jsou vytvořeny nové návrhy,prováděny změny v aktuálních návrzích, změna aplikace stávajících návrhů nebo zlepšení stávajích návrhů. Samozřejmě se může jednat i o celé systémy, procesy, služby atd. Přesto, že je FMEA prováděna před dokončením nebo při dokončení realizace dané fáze vývoje, je dobré přistupovat k dokumentům obsahujícím analýzu jako k dynamickým a v čase zlepšovat jejich kvalitu a přesnost. \cite{fmeatheory} Zaměřením se analýza člení zejména na tři typy, které se odvíjejí podle fáze vývoje. Jedná se o tyto tři druhy: 

\begin{itemize}
	\item  DFMEA (Design)
	\item  PFMEA (Process)
	\item  SFMEA (System)
\end{itemize}

V této diplomové práci bude věnována pozornost hlavně analýze zaměřené na návrh a proces. Tyto dvě zaměření jsou v praxi využívány nejvíce pravděpodobně díky tomu, že se zaměřují na specifičtější části vývoje produktu. Součástí FMEA analýzy je dekompozice produktu ve fázi strukturální analýzy na nízkoúrovňové části, pro které se lépe hledají možné způsoby selhání. Použití metody na celý systém se tak může zdát jako zbytečně komplexní řešení a v některých případech tato možnost ani nepřipadá v úvahu. Konkrétně je dobré uvést příklad z rozsáhlého automobilového průmyslu, kde se výroba vozu skládá ze spolupráce několika různých výrobců komponent, kteří často nemusí mít přehled o práci ostatních dodavatelů. V některých případech jsou výrobky součástí většího celku a těžko se při použití metody DFMEA nalézají všechny možné rizikové scénáře. V těchto případech se například provádí pouze analýza zaměřená na konkrétní výrobní proces, u kterého lze určit možná rizika. 

\subsection{DFMEA}
\label{subsec:DFMEA}
DFMEA se používá pro analýzu nových návrhů produktů. \cite{dfmea} Měla by tedy navazovat na ukončení fáze návrhu a vycházet tak ze softwarových artefaktů z toho vyplývajících. Jedním z těchto artefaktů může být například blokový diagram(Boundary diagram), který určuje rozsah analyzované části systému. Pomocí blokového diagramu lze také zobrazit rozhraní a vztahy mezi jednolivými komponentami. Mezi další podklady pro tvorbu DFMEA mohou patřit: 
\begin{itemize}
	\item  Normy a předpisy
	\item  Fyzikální specifikace materiálů
	\item  Návrhy obdobných produktů, popř. předchozí DFMEA, pokud byla provedena
	\item  Požadavky zákazníka
	\item  Diagram parametrů (P-diagram)
	\item  Shrnutí všech požadavků na návrh
\end{itemize}



\subsection{PFMEA}
\label{subsec:PFMEA}
PFMEA se používá pro analýzu nových výrobních procesů. \cite{pfmea} Dá se říci, že by měla navazovat na předchozí analýzu návrhu produktu, kdy výrobní proces je dalším logickým krokem při vývoji. Mezi vstupní podklady patří například vývojový diagram procesu, který dekomponuje proces do série navazujících kroků. Mezi další podklady pro tvorbu PFMEA mohou patřit: 
\begin{itemize}
	\item  DFMEA
	\item  Specifické zákaznické požadavky
	\item  Zákonné požadavky a předpisy
	\item  Matice znaků
	\item  Zkušenosti z předešlého vývoje podobného procesu
\end{itemize}
V praxi je možné setkat se i s případem, kdy se PFMEA provádí až jako snaha o zlepšení stávajícího procesu. V tomto případě lze jako podklad pro vypracování také vzít v úvahu fyzickou prohlídku výrobní haly. Tuto analýzu často vykonává stejný tým jako v případě DFMEA. Výstupem by měl být tzv. kotrolní plán, který obsahuje seznam konkrétních změn, které by měly být provedeny na základě provedené analýzy. 
\endinput