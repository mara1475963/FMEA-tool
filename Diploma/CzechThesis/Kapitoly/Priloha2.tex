\chapter{Instalace webové aplikace}
\label{sec:docker}
Pro nasazení alikace je použit docker, což je software, který umožňuje běh aplikace v izolovaném prostředí tzv. kontejneru. Díky dockeru je možné spustit stáhnutý projekt použitím příkazu \uv{docker-compose up} v kořenovém adresáři projektu. Zde se nachází soubor s příponou yaml, který obsahuje skript pro vytvoření obrazu pro klientskou a serverovou část a spuštění aplikace na adrese: http://localhost:3000/. Pro otestování sdílení dat mezi uživateli je možné otevřít stránku se stejným url ve dvou separátních oknech. Díky většímu množství použitých knihovena a závislostí může prvotní instalace a spuštění trvat i několik minut. Docker konkrétně usnadňuje uživateli nutnost použití několika příkazů pro instalaci všech potřebných závislostí a spuštění serverové a klientské části. Tyto příkazy jsou obsaženy v souborech s názvem Dockerfile umístěných v adresářích /frontend a /backend. Standardně se řeší také inicializace a spuštění databáze. Nicméně toto bylo vyřešeno použitím cloudového řešení, ke kterému je možné se připojit z jakékoliv IP adresy. Běh aplikace v kontejneru je možné ukončit příkazem \uv{docker-compose down}.
