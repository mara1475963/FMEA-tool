\chapter{Úvod}
\label{sec:Uvod}
V mnoha odvětvích lidské činnosti je možné přijít do styku s nedokonalostmi v návrzích produktů nebo výrobních procesech, které mohou vést k nezamýšlenému chování a více či méně važným následkům. Následky mohou být triviální, které výrazným způsobem neomezují funkci produktu nebo naopak vážné, které mohou být životu nebezpečné. Dalším důsledkem špatného technologického procesu nebo návrhu je samozřejmě také finanční stránka, kdy nadměrná produkce disfunkčních dílů stojí výrobce peníze navíc za materiál nebo při objevení závady až u koncové zákazníka také výdaje s vyřizováním reklamací, opravami a podobně. Tyto důvody vedly k vytvoření procesů, metod a norem, které mají za cíl odhalení, odstranění nebo alespoň zmírnění možných závad a rizik ještě před začátkem realizace dané činnosti. Souhrně lze tyto snahy označit Risk Management, tedy disciplínu zabývající se správou a řízením rizik. 

Jednou z těchto metod je analytická metoda FMEA(Failure Mode and Effects Analysis), kterou se zabývá tato diplomová práce. Následující kapitola  \ref{sec:FMEA} bude obsahovat základní popis této metody spolu s její historií, dále popisem v jakých oborech se metoda upltňovala nebo aktuálně nejvíce uplatňuje a také rozdělení na základní typy podle toho v jaké fáze vývoje produktu se analýza provádí.

Kapitola \ref{sec:FMEA_postup} se budou zabývat tím, jaký je postup při vypracovávání této analýzy. Analýza se skládá celkem ze sedmi kroků, které budou podrobně vysvětleny, budou zde také uvedeny výňatky tabulky z formulářů, ve kterých se analýza provádí. Na jednotlivých krocích analýzy budou zobrazeny odlišnosti dvou typů analýz, na které je kapitola zaměřena.

Následovat bude kapitola \ref{sec:pozadavky} sloužící jako popis požadavků na vytvářený nástroj. Specifikace požadavků bude inspirována disciplínou inženýrství požadavků. Požadavky budou obsahovat zařazení dle typů a kapitol, popis a také jejich konečný stav, podle toho jak se vyvýjel v rámci průběhu návrhu a implementace. Tento seznam požadavků vychází jak ze samotného zadaní diplomové práce, tak z konzultací s vedoucím práce a také rešerše existujících řešení, kterou se bude zabývat kapitola \ref{sec:nastroje}. 

Rešerše existujících řešení je cíleně umístěna až za kapitolou zabývající se analýzou a specifikací požadavků z toho důvodu, že budou v této kapitole srovnány jednotlivé nástroje na základě sady požadavků vzniklé v předchozí kapitole. Bude se jednat o zjedonodušenou verzi požadavků. 
\endinput