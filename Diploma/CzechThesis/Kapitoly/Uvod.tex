\chapter{Úvod}
\label{sec:Uvod}
V mnoha odvětvích lidské činnosti je možné přijít do styku s nedokonalostmi v návrzích produktů nebo výrobních procesech, které mohou vést k nezamýšlenému chování a více či méně važným následkům. Následky mohou být triviální, které výrazným způsobem neomezují funkci produktu nebo naopak vážné, které mohou být životu nebezpečné. Dalším důsledkem špatného technologického procesu nebo návrhu je samozřejmě také finanční stránka, kdy nadměrná produkce disfunkčních dílů stojí výrobce peníze navíc za materiál nebo při objevení závady až u koncové zákazníka také výdaje s vyřizováním reklamací, opravami a podobně. Tyto důvody vedly k vytvoření procesů, metod a norem, které mají za cíl odhalení, odstranění nebo alespoň zmírnění možných závad a rizik ještě před začátkem realizace dané činnosti. Souhrně lze tyto snahy označit Risk Management, tedy disciplínu zabývající se správou a řízením rizik. 

Jednou z těchto metod je analytická metoda FMEA, kterou se zabývá tato diplomová práce. Následující kapitola  \ref{sec:FMEA} bude obsahovat základní popis této metody spolu s její historií a také rozdělením na základní typy podle toho v jaké fázi vývoje produktu se analýza provádí.

Kapitola \ref{sec:FMEA_postup} se budou zabývat tím, jaký je postup při vypracování této analýzy. Analýza se skládá celkem ze sedmi kroků, které budou podrobně vysvětleny. Budou zde také uvedeny výňatky formulářů z tabulky, které se používají pro její tvorbu. Na jednotlivých krocích budou zobrazeny odlišnosti dvou typů FMEA, na které je práce zaměřena.

Následovat bude kapitola \ref{sec:pozadavky} sloužící jako popis požadavků na vytvářený nástroj. Specifikace požadavků bude inspirována disciplínou inženýrství požadavků. Požadavky budou členěny dle typů a kapitol, budou obsahovat popis a také jejich konečný stav, podle toho jak se vyvýjel v rámci průběhu návrhu a implementace. Některé zajímavé požadavky budou popsány detailněji z důvodu vysvětlení důvodu jejich zařazení nebo naopak nezařazení do celkové sady požadavků na vyvýjený nástroj. Tento seznam požadavků vychází jak ze samotného zadaní diplomové práce, tak z konzultací s vedoucím práce a také rešerše existujících řešení, kterou se bude zabývat kapitola \ref{sec:nastroje}. 

Rešerše existujících řešení je cíleně umístěna až za kapitolou zabývající se analýzou a specifikací požadavků z toho důvodu, že budou v této kapitole srovnány jednotlivé nástroje na základě zjednodušené sady požadavků vzniklé v předchozí kapitole. Součástí srovnání bude i nástroj vytvořený autorem této práce, kde bude vidět srovnání, jak byly jednotlivé požadavky naplněny oproti profesionálním řešením dosupným na trhu. 

Další kapitola číslo \ref{sec:navrh} se již bude zabývat vlastním nástrojem pro tvorbu FMEA, který byl vytvořen v rámci této diplomové práce. Konkrétně bude obsahovat popis architektury, použitých technologií, návrh databáze a uživatelského rozhraní. Budou zde uvedeny i důvody proč byly jednotlivé návrhové rozhodnutí učiněny. 

Popisem implementace vytvořeného nástroje se bude zabývat kapitola \ref{sec:implementace}. Implementace bude rozdělena na klientskou a serverovou část. Popis bude brán jak z uživatelského, tak implementačního hlediska. Budou zde podrobněji ukázány jednotlivé vlastnosti a funkce nástroje. Konec kapitoly se bude zabývat testováním vytvořeného nástroje ve formě systémových testů. 

Závěrem bude hodnocení dosažených výsledků a vytvořeného nástroje. Budou zde také uvedeny různé možnosti rozšíření nástroje, které z různých důvodu nebyly implementovány. 
\endinput