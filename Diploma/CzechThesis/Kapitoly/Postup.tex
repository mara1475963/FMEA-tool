\chapter{Vypracování FMEA }
\label{sec:FMEA_postup}
V této kapitole bude podrobněji popsán postup vypracování metody FMEA pro návrh a proces. Obě tyto varianty mají totožnou strukturu a tak budou její jednotlivé části popsány zároveň pro vytvoření obrazu v jakých atributech se konkrétní části obou typů analýz liší. 

Výňatky z formulářů odpovídají standardu příručky FMEA pro automobilový průmysl. \cite{fmeaHandbook} Jednotlivé reprezentace FMEA se mohou lišit, zejména při použití v odlišných odvětvích. Jak již bylo naznačeno v kapitole \ref{sec:historie}, tak lze analýzu aplikovat v odvětvích jako je armádní průmysl, zdravotnictví, kosmický průmysl a mnoho dalších. Nicméně nejrozšířenějším odvětvím, který využívá metody FMEA je automobilový průmysl a i proto budou uvedené příklady z tohoto odvětví. \cite{fmeaExamples} Také je možné se setkat s některými volitelnými atributy, které nemají veliký význam a nebude na ně brán v následujícím popisu zřetel.  

\section{Plánování a příprava(1. krok)}
V kapitolách \ref{subsec:DFMEA} a \ref{subsec:PFMEA}  byly zmíněny podklady, které je dobré zajistit před začátkem analýzy. V prvním kroku je také sestaven tým odborníků, kteří budou za pomocí společných setkání provádět samotnou analýzu. V týmu je potřeba před začátkem práce definovat základní pravidla a slovník pojmů. Vhodné je také definovat vlastní škálu pro určení hodnotících atributů. Příkladem může být určení vlastních příkladů pro danou číselnou hodnotu atributu, určení priority daných atributů nebo nastavení hranic pro určení míry rizika. Jedná se o tyto hodnotící atributy: 

\begin{enumerate}
	\item \textbf{Severity} (Vážnost) - závažnost selhání
	\item \textbf{Occurance} (Výskyt) - frekvence selhání
	\item \textbf{Detection} (Detekce) - schopnost detekovat selhání než se objeví u zákazníka
\end{enumerate}

Pro každý z těchto tří atributů je dobré stanovit škálu, která nabývá hodnot od jedné do desíti a každé číselné hodnocení obsahuje slovní popis této hodnoty pro jednodušší určení hodnoty. Součástí nástrojů  bývají tabulky, které obsahují i základní slovní hodnocení jednotlivých hodnot atributů. Po provedení všech přípravných akcí je možné přistoupit k vyplnění informací tvořící hlavičku analýzy. 

\begin{center}
\begin{table}[h]
	\centering
	\caption{Hlavička analýzy}
 \label{tab:header_FMEA}
	\label{tab:Head1}
        \begin{tabular}{|c | r | l |} 
         \hline
 \multicolumn{3}{|c|}{Planning and preparation (Step 1)} \\

         \hline
         1 & Company Name & ABC  \\ [0.5ex] 
         \hline
         2 & Location & Bratislava  \\ [0.5ex] 
         \hline
         3 & Customer Name & VW Group  \\ [0.5ex] 
         \hline
         4 & Model Year / Program(s) & 2006/J77 \\ [0.5ex] 
         \hline
         5 & Subject & J77 Instrument Cluster \\ [0.5ex] 
         \hline
         6 & Responsibility & John Doe (Product Eng.) \\ [0.5ex] 
         \hline
         7 & D/PFMEA Start Date & 1.1.2023  \\ [0.5ex] 
         \hline
         8 & D/PFMEA Revision Date & 27.2.2023 \\ [0.5ex] 
         \hline
         9 & D/PFMEA ID Number & 77 \\ [0.5ex] 
         \hline
         10 & Confidentiality Level & Proprietary \\ [0.5ex] 
         \hline
         11 & Cross-Functional team & John Doe, Henry Smith \\ [0.5ex] 
         \hline
        \end{tabular}
    \end{table}
\end{center}

Tabulka \ref{tab:header_FMEA} obsahuje seznam atributů, které se vyplňují v rámci úvodní fáze analýzy. Zde je popis jednotlivých atributů:

\begin{enumerate}
	\item \textbf{Název společnosti} - Jedná se o společnost v rámci, které je analýza prováděna.
	\item \textbf{Lokace} - Lokace může odpovídat buď umístěním inženýrského týmu zodpovědného za návrh produktu(DFMEA) nebo umístěním výrobního závodu(PFMEA). 
	\item \textbf{Jméno zákazníka} - Podle zaměření analýzy, může být zákazníkem jak koncový uživatel, tak i interní oddělení, které navazuje ve výrobním cyklu.
 \item \textbf{Modelový rok, Program} - Specifikace modelu produktu, programu(DFMEA) nebo platformy(PFMEA).
 \item \textbf{Předmět} - Označení zkoumaného prvku.
 \item \textbf{Zodpovědnost} - Vlastník dokumentu analýzy, často v roli manažera setkání týmu.
 \item \textbf{Začátek analýzy} - Počáteční datum.
 \item \textbf{Datum revize} - Datum kontroly provedených změn.
 \item \textbf{Identifikační číslo analýzy} - Jedná se o interní hodnotu, má smysl v rámci kontextu firmy, kdy může sloužit jako odkaz na dokument s analýzou.
 \item \textbf{Stupeň důvěrnosti} - Může nabývat hodnot obchodní, proprietární a důvěrný.
 \item \textbf{Multioborový tým} - Seznam lidí podílejících se na tvorbě analýzy.
\end{enumerate}

\section{Analýza struktury(2. krok)}
\label{sec:FMEA_postup_2}
Cílem strukturální analýzy je dekompozice zkoumaného návrhu nebo procesu na menší části, které pak slouží jako vstupní parametry v dalších krocích. Dekompozice slouží pro jednodušší pochopení zkoumaného prvku. Výsledkem je stromová struktura o výšce stromu tři, kdy kořenem stromu je přímo zkoumaný prvek.

\begin{center}
\begin{table}[h]
	\centering
	\caption{Formulář pro analýzu struktury(DFMEA) }
	\label{tab:structure_DFMEA}
\begin{tabular}{ |p{5cm}|p{5cm}|p{5cm}|  }
 \hline
 \multicolumn{3}{|c|}{Structure Analysis (Step 2)} \\
 \hline
 1. Next Higher Level& 2. Focus Element
No. And Name of Focus Element &3. Next Lower Level of Characteristic Type\\
 \hline
 Instrument Cluster B90   & 6. Speed clock    &Stepper motor\\


 \hline
\end{tabular}\  
\end{table}
\end{center}

\begin{center}
\begin{table}[h]
	\centering
	\caption{Formulář pro analýzu struktury(PFMEA) }
	\label{tab:structure_PFMEA}
\begin{tabular}{ |p{5cm}|p{5cm}|p{5cm}|  }
 \hline
 \multicolumn{3}{|c|}{Structure Analysis (Step 2)} \\
 \hline
1. Process Item
System, Subsystem, Part Element or Name of Process
& 2. Process Step
No. And Name of Focus Element
&3. Process Work Element

4M Type\\
 \hline
X98 sun roll assembly line   & 020 Assembly body-mechanism   &Operator\\


 \hline
\end{tabular}\  
\end{table}
\end{center}

V tomto kroku analýzy se nejvíce projevují rozdíly mezi jejími dvěma zmíněnými typy. Rozdílnost atributů, které jsou v této fázi vyplňovány je možné porovnat v tabulkách \ref{tab:structure_DFMEA} a  \ref{tab:structure_PFMEA}. 
Dále následuje popis jednotlivých atributů, které se vyplňují v rámci strukturální analýzy. 

\begin{enumerate}
	\item \textbf{Vyšší úroveň / Položka procesu systému, subsystému, dílu nebo název procesu} - Prvek na nejvýšší úrovni v rámci předmětu analýzy, pro tento prvek se v se v rámci analýzy selhání definují důsledky selhání.
	\item \textbf{Vybraný prvek, číslo a označení/Krok procesu, číslo a označení} - Prvek na druhé úrovni, pro tento prvek se v rámci analýzy selhání určuje jeden ze základních pílířů analýzy, a to způsob selhání.
	\item \textbf{Nižší úroveň nebo druh charakteristiky/Prvek provádění činnosti, 4M} - Prvek na nejnižší úrovni, pro který se určují příčiny selhání. V analýze zaměřené na proces je zde uveden návodně typ 4M, který vyjadřuje(Machine, Man, Material, Milieu), tedy určité kategorie, do kterých by mohl hodnotící tým zařadit prvek na této úrovni. Další kategorie mohou také být Method, Measurement.  
\end{enumerate}

\section{Analýza funkcí(3. krok)}
\label{sec:FMEA_postup_3}
V rámci analýzy struktury byly vytvořeny objekty, pro které ve fázi analýza funkcí bude tým definovat jejich funkce. Nalezené funkce budou dále sloužit jako vstupní parametry pro analýzu selhání. Stejně jako v předchozím kroku, kdy výsledná analýza tvořila stromovou strukturu, neboli jednotlivé objekty měli mezi sebou definované relační vazby, tak je tomu obdobně i v této části analýzy. Zde je kladen důraz hlavně na funkcionalitu prvků na druhé úrovni. Pro tyto funkce se pak hledají funkce ze třetí úrovně, které zajišťují jejich funkcionalitu. Stejně je potřeba nalézt funkce na první úrovni, které jsou chápany jako výsledky funkce ze druhé úrovně. Analýza funkcí je dalším příkladem, kdy se jednotlivé typy analýzy zaměřené na návrh a proces liší. Tyto odlišnosti znázorňují tabulky \ref{tab:function_DFMEA} a \ref{tab:function_PFMEA}.

\begin{center}
\begin{table}[h]
	\centering
	\caption{Formulář pro analýzu funkcí(DFMEA) }
	\label{tab:function_DFMEA}
\begin{tabular}{ |p{5cm}|p{5cm}|p{5cm}|  }
 \hline
 \multicolumn{3}{|c|}{Function Analysis (Step 3)} \\
 \hline
 1. Next Higher Level Function and Requirement &
2. Focus Element
Function and Requirement &
3. Next Lower Level Function and Requirement or Characteristic\\
 \hline
 Display vehicle status: speed, torq, gas level, engine temperature,..   & Display vehicle speed    & Convert a train of input pulses into a precisely defined increment in the shaft position\\


 \hline
\end{tabular}\  
\end{table}
\end{center}

\begin{center}
\begin{table}[h]
	\centering
	\caption{Formulář pro analýzu funkcí(PFMEA) }
	\label{tab:function_PFMEA}
\begin{tabular}{ |p{5cm}|p{5cm}|p{5cm}|  }
 \hline
 \multicolumn{3}{|c|}{Function Analysis (Step 3)} \\
 \hline
1. Function of the Process Item
Function of System, Subsystem, Part Element or Name of Process
& 2. Function of the Process Step and Product Characteristics
(Quantitative value is optional)
& 3. Function of the Process Work Element and Process Characteristics
\\
 \hline
Assembly body with mechanism   & Screw body with mechanism with 2 screws   &Take the first screw and insert into driver then press tool start button\\


 \hline
\end{tabular}\  
\end{table}
\end{center}

\newpage
Popis atributů v rámci analýzy funkcí:
\begin{enumerate}
	\item \textbf{Funkce a požadavek vyšší úrovně / Funkce položky procesu Funkce systému, subsystému, dílu/komponentu nebo procesu} - Funkce odpovídající prvku nebo procesu na nejvyšší úrovni. Tomuto atributu může být přiřazeno více hodnot, ne pouze jedna jako v případě analýzy struktury.
	\item \textbf{Funkce a požadavek vybraného prvku/Funkce procesního kroku a charakteristika produktu(kvantitativní hodnota je volitelná)} - Funkce odpovídající prvku na druhé úrovni(subsystém, komponent) nebo kroku procesu(čeho musí daná stanice dosáhnout).
	\item \textbf{Funkce a požadavek nebo charakteristika nižší úrovně/Funkce prvku provádějící činnost a charakteristika procesu} - Funkce odpovídající prvku na nejnižší úrovni(komponent, díl) nebo prvku provádějící nějakou činnost v procesu.
\end{enumerate}

\section{Analýza selhání(4. krok)}
Analýza selhání je klíčovým krokem v rámci FMEA. Z předchozí činnosti, kdy byly objektům přiřazeny funkce, je cílem tohoto kroku zjistit, jak by objekt o dané funkci mohl selhat. Pro nalezené selhání se dále hledají důsledky a příčiny, které jsou přiřazeny funkcím na první a druhé úrovni v rámci celkové struktury. Stejně jako u předchozích dvou kroků existují mezi atributy relační vazby. Zde je celkem logicky každému selhání přiřazeno několik možných důsledků a příčin. Důsledky selhání jsou definovány jako dopad na celkový systém, a proto náleží kořenovému objektu a nějakým jeho funkcím. Příčiny selhání jsou definovány jako selhání prvků na nejnižší úrovni dekompozice a slouží spolu s ostatními atributy jako vstupní parametry pro analýzu rizik v následujícím kroku. Analýza selhání je krok, při kterém se již dva zmiňované typy od sebe tolik neliší. Popis atributů je zde podobný, nicméně se odkazují na různé funkce z předchozího kroku a může se lehce lišit i přístup pro definování jednotlivých selhání. Porovnání obou typů zobrazují tabulky \ref{tab:failure_DFMEA} a \ref{tab:failure_PFMEA}.

\begin{center}
\begin{table}[htp]
	\centering
	\caption{Formulář pro analýzu selhání(DFMEA) }
	\label{tab:failure_DFMEA}
\begin{tabular}{ |p{5cm}|p{0.5cm}|p{4cm}|p{4cm}|  }
 \hline
 \multicolumn{4}{|c|}{Failure Analysis (Step 4)} \\
 \hline
  1. Failure Effects (FE) to the next higher level and/or End User&
  \begin{turn}{-90}Severity (S) of FE\end{turn} &
2. Failure Mode (FM) of the Focus Element &
3. Failure Cause (FC) of the Next Lower Level Element or Characteristic
\\
 \hline
 Speed not display.
  & 5
  & Speedometer needle lock to 0 position. & Lost connection between stepper motor and board due bad soldering.
\\
 \hline
\end{tabular}\  
\\
\hfill \break
\hfill \break
\centering
	\caption{Formulář pro analýzu selhání(PFMEA) }
	\label{tab:failure_PFMEA}
\begin{tabular}{ |p{5cm}|p{0.5cm}|p{4cm}|p{4cm}|  }
 \hline
 \multicolumn{4}{|c|}{Failure Analysis (Step 4)} \\
 \hline
  1. Failure Effects (FE)
&
  \begin{turn}{-90}Severity (S) of FE\end{turn} &
2. Failure Mode (FM) of the Process Step
 &
3. Failure Cause (FC) of the Work Element

\\
 \hline
\textbf{Plant:}
Gap between housing and mechanism more than tolerance - rework necessary (less than 100% of batch) 

\textbf{Ship to Plant:}
Additional insertion force to assembly it into door panel

\textbf{End user:}
Possibility in time to hear noises (disturbed)

  & 5
  & Missing one screw
 & Operator skip

\\
 \hline
\end{tabular}\  
\end{table}
\end{center}



\clearpage
Popis atributů v rámci analýzy selhání:
\begin{enumerate}
	\item \textbf{Důsledky selhání týkající se prvku na  nejvyšší úrovni nebo koncového uživatele } -  Jak je vidět na uvedeném příkladu pro PFMEA, tak je možné držet se určitých mantinelů pro definování dopadů selhání. Konkrétně je možné určit dopady na vlastní továrnu, továrnu zákazníka nebo koncového uživatele. Ve formulářích pro analýzu selhání je uveden i atribut Vážnost, který se ale hodnotí až v rámci analýzy rizik. Proto mu bude věnováno více pozornosti až v kapitole \ref{sec:FMEA_postup_5}.
	\item \textbf{Selhání vybraného prvku/kroku procesu} - Selhání(vada) prvku nebo procesního kroku na druhé úrovni. Jedná se o hlavní atribut v rámci celé analýzy. Tým stanovující  selhání může vycházet například z různých předpokladů nebo obdobných systému a jejich FMEA(pokud byla provedena). Stanovení hodnocení tohoto atributu je důležité i z hlediska názvosloví a je potřeba používat odbornou a ucelenou terminologii.  
	\item \textbf{Přičina selhání prvku nižší úrovně nebo charakteristiky/Přičina selhání prvku provádějící činnost v rámci procesu} - Přičina selhání odpovídající prvku na nejnižší úrovni. V případě PFMEA odpovídá prvku z kategorie 4M, kdy může být prvkem osoba provádějící danou činnost. 
\end{enumerate}


\section{Analýza rizik(5. krok)}
\label{sec:FMEA_postup_5}
Analýza rizik vychází z předchozího kroku, kdy hodnotící tým stanovil možné způsoby selhání, jejich příčiny a důsledky. V rámci analýzy tohoto kroku bude tým odborníků hodnotit toto riziko na základě aktuálně stanovených opatření. Výstupem hodnocení pro každé riziko musí být jedna z následujích variant:
\begin{itemize}
	\item Aktuální opatření jsou dostačující/hodnocení rizika není na tolik závažné, aby bylo nutné přistupovat k nějakým změnám. 
    \item Riziko může mít negativní dopady do té míry, že je potřeba závést nové opatření nebo upravit stávající tak, aby došlo ke zmírnění hodnocení některých atributů.
\end{itemize}
Analýza jako celek nemůže být uzavřena, dokud nejsou všechny rizika v jednom z uvedených stavů. Tato fáze analýzy je pro oba typy FMEA prakticky totožná a význam atributů je stejný, proto bude uveden pouze jeden společný příklad(viz. tabulka \ref{tab:risk_FMEA}).

\begin{center}
\begin{table}[h]
	\centering
	\caption{Formulář pro analýzu rizik }
	\label{tab:risk_FMEA}
\begin{tabular}{|p{6cm}|p{0.5cm}|p{6cm}|p{0.5cm}|p{0.5cm}|  }
 \hline
 \multicolumn{5}{|c|}{Risk Analysis (Step 5)} \\
 \hline
1. Current preventive control (PC) for FC
&
  \begin{turn}{-90}2. Occurence (O) of FC\end{turn} &
3. Current detection control (DC) for FC or FM
 &
  \begin{turn}{-90}4. Detection (D) of FC or FM\end{turn}
 &
  \begin{turn}{-90}5. D(P)FMEA AP\end{turn}

\\
 \hline
No prevention.
& 10
& Testing method to be developed.
& 10
& H


\\
 \hline
\end{tabular}\  
\end{table}
\end{center}

\newpage
Popis atributů v rámci analýzy rizik:
\begin{enumerate}
	\item \textbf{Stávající preventivní opatření k příčinám} - Příklad opatření, které firma aktuálně využívá a slouží jako prevence selhání. 
	\item \textbf{Výskyt(frekvence) příčiny} - Jeden ze tří hodnotících atributů. Nabývá hodnot ze stupnice 1-10. Číselnému hodnocení odpovídá také hodnocení slovní, které usnadňuje stanovení číselné hodnoty. Ač je definice tohoto atribut pro oba typy analýzy podobná, tak se používá pro určení hodnoty odlišná tabulka. Jeden z aspektů pro hodnocení je také předchozí atribut. Dá se říci, že čím kvalitnější jsou stávající preventivní opatření, tím bude hodnota tohoto atributu menší.     
	\item \textbf{Stávající opatření pro detekci příčiny nebo selhání} - Metoda, kterou tým v rámci tohoto atributu určuje již není preventivní, ale slouží k odhalení vzniklého selhání. 
	\item \textbf{Detekce příčiny} - Poslední hodnotící atribut. Stejně jako u atributu výskytu, tak dosahuje hodnot 1-10, kterým odpovídá i slovní popis. Taktéž navazuje na předchozí atribut, který také hraje roli v tom, jak moc vysoká bude jeho číselná hodnota. 
	\item \textbf{Priorita v rámci DFMEA nebo PFMEA} - 
 Ještě před popisem posledního atributu je potřeba zmínit atribut Severity(Vážnost), který byl součástí předchozí ukázky formuláře v tabulce \ref{tab:failure_DFMEA} nebo \ref{tab:failure_PFMEA}. Vážnost je první z hodnotících atributů, který má obvykle v rámci celkového hodnocení rizika největší váhu. Atribut se váže na množinu dopadů odpovídající nějakému selhání, tzn. že pro více hodnot Výskyt a Detekce může být použita společná hodnota atributu Význam. Hodnocení tohoto atributu probíhá stejným způsobem jako u Výskytu a Detekce, nicméně tento atribut vyjadřuje i určité dopady selhání nebo příčinny na koncového zákazníka. Tyto dopady mohou být triviální, kdy dochází například pouze k ovlivnění vzhledu, zvuku nebo vibracím, ale také velice závažným, kdy může být uživatel ohrožen na životě. Příkladem takové situace je, když je analyzovaným produktem nějaký důležitý komponent automobilu. 
 
 Poslední atribut analýzy rizik slouží jako součást rozhodnutí, jak s odhaleným rizikem naložit. Tento atribut vychází ještě z jedné hodnoty, která se nicméně již v nových verzích analýzy neuvádí. Touto hodnotou je Risk Priority Number a vyjadřuje součin číselného hodnocení atributů Vážnost, Výskyt a Detekce. Logicky se bude tato hodnota pohybovat v rozmezí 1-1000. O výpočet této hodnoty se z pravidla stará použitý software. Na základě vypočtené hodnty RPN je stanovena i hodnota atributu AP, která může nabývat hodnot high, medium a low. Jak již bylo řečeno AP je pouze součástí pro rozhodnutí do jakého stavu dané riziko přejde. Tím, že hodnota AP vychází ze součinu, jehož tři činitelé mají teoreticky stejnou váhu, tak nemusí hodnota plně reflektovat stav daného rizika. Pro ucelené a konečné hodnocení je potřeba v rámci týmu vzít v úvahu i konkrétní situaci a váhu stanovených atributů.
 
\end{enumerate} 
\section{Optimalizace(6. krok)}
\label{sec:FMEA_postup_6}
Ná základě předchozího kroku tým vykonávající analýzu vyhodnotil pro jaké rizika je potřeba zavést dodatečné opatření pro zmírnění některého z atributů a snížení celkového hodnocení v rámci AP. Těchto rizik se bude týkat předposlední krok a to je Optimalizace. Zjednodušeně je cílem tohoto kroku stanovit nové opatření pro prevenci a detekci selhání, přiřadit provedení změn konkrétní osobě, určit datum revize a po uplynutí tohoto data provést opětovné hodnocení atributů Vážnost, Výskyt a Detekce s předpokladem, že bude výsledná hodnota rizika zmírněna. Tento krok je posledním v rámci formuláře, který má skupina hodnotitelů k dispozici. Stejně jako tomu bylo u několika předchozích kroků, tak je význam atributů v tomto kroku společný pro oba typy analýzy. Dále následuje tabulka \ref{tab:optimization_FMEA}, která obsahuje seznam všech atributů, které se v rámci tohoto kroku vyplňují.

\begin{sidewaystable}
	\centering
	\caption{Formulář pro Optimalizaci }
	\label{tab:optimization_FMEA}
\begin{tabular}{|p{2.5cm}|p{2.5cm}|p{2.5cm}|p{2cm}|p{0.5cm}|p{2cm}|p{2.5cm}|p{0.5cm}|p{0.5cm}|p{0.5cm}|p{0.5cm}| }
 \hline
 \multicolumn{11}{|c|}{Optimization (Step 6)} \\
 \hline
1. D(P)FMEA Prevention Action &
2. D(P)FMEA Detection Action &
3. Responsible Persons Name &
4. Target Completion Date &
\begin{turn}{-90}5. Status\end{turn} &
6. Action taken with Pointer to Evidence &
7. Completion Date &

\begin{turn}{-90}8. Severity(S)\end{turn} &
\begin{turn}{-90}9. Occurence (O)\end{turn} &
\begin{turn}{-90}10. Detection (D)\end{turn} &
\begin{turn}{-90}11. D(P)FMEA AP\end{turn}

\\
 \hline
Re-use same design as for X44 project - proved solution. &
Update test method - test to failure method.& 
John Doe & 
15.05.2020 &
C & 
Design Review 244243 May 27 2020 & 
15.05.2020 & 
5 & 
3 & 
3 & 
L 

\\
 \hline
\end{tabular}
\end{sidewaystable}

\newpage
Popis atributů v rámci optimalizace:
\begin{enumerate}
	\item \textbf{Preventivní opatření} - Zavedení nového preventivního opatření nebo úprava stávajícího.
	\item \textbf{Opatření k odhalení} - Zavedení nového opatření pro detekci chyby nebo úprava stávajícího.
	\item \textbf{Odpovědná osoba} - Přiřazení realizace změn v rámci opatření konkrétní osobě. Tato osoba by měla být součástí týmu a také uvedena ve skupině účastníků v hlavičce dokumentu.
	\item \textbf{Plánované datum dokončení}
	\item \textbf{Status} - Tento atribut udává v jakém stavu je realizace změn v opatřeních. Může nabývat například hodnot (O - Open, DP - Decision Pending, IP - Implementation Pending, C - Completed, NP - Not Implemented)
	\item \textbf{Přijatá opatření s odkazem na důkaz} - Dokumentace provedených změn. Měl by obsahovat i odkaz na další dokument popisující provedené opatření např. kontrolní plán v rámci PFMEA.
	\item \textbf{Datum dokončení} 
	\item \textbf{Vážnost} - Opětovné hodnocení atributu Vážnost
	\item \textbf{Výskyt} - Opětovné hodnocení atributu Výskyt
	\item \textbf{Detekce} - Opětovné hodnocení atributu Detekce
	\item \textbf{Priorita} - Po stanovení všech nových opatření se provádí reevaluace všech hodnotících atributů. Cílem je zjistit, jestli byla provedená opatření natolik účinná, že je možné ohodnotit jednotlivé atributy  nižším číslem a tím také snížit prioritu rizika. 
\end{enumerate}

\section{Dokumentace výsledků(7. krok)}
Dokumentace výsledků je posledním krokem v rámci vypracování FMEA. Jeho účelem je sumarizace dosažených výsledků a provedených zlepšení v rámci návrhu produktu nebo výrobního procesu. Jak již bylo řečeno předpokladem pro provedení dokumentace je to, že byly uzavřeny všechny nalezené rizika. Dokumentace výsledků již není součástí vyplňovaného formuláře. Dokument může mít libovolný formát vyhovující potřebám firmy. FMEA může sloužit i jako záruka kvality při komunikaci se zákazníkem v roli obchodního řetězce prodávájící daný produkt. Pro tyto účely je možné také použít toto zhodnocení vytvořené v rámci tohto kroku. Zhodnocení také může sloužit pro informování vysoko postavených osob ve vedení firmy o stavu produktu v rámci kontroly kvality. 
  