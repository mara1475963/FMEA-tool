\chapter{Vypracování FMEA analýzy}
\label{sec:FMEA_postup}
V této kapitole bude podrobněji popsán postup při použití metody FMEA pro návrh a proces. Obě tyto varianty mají totožnou strukturu a tak budou jednotlivé její části popsány zároveň pro vytvoření obrazu v jakých atributech se konkrétní prvky budou lišit. Názorný příklad bude ze světa automobilismu. 

\section{1. krok Plánování a příprava}
V kapitolách na DFMEa a PFMEa byly zmíněny předpoklady a vstupní podklady před začátkem samotné analýzy. V tomto kroku je také sestaven tým odborníků, kteří budou za pomocí společných setkání provádět samotnou analýzu. V tomto týmu je potřeba před začátkem práce definovat základní pravidla a pojmy. Je potřeba definovat škálu pro atributy, které pomocí kterých bude probíhat hodnocení nalezených rizik. Od této škály se například bude odvíjet, od jaké hodnoty bude akceptována určitá míra rizika. Tyto atributy jsou: 

\begin{enumerate}
	\item Severity (Vážnost)
	\item Occurance (Výskyt)
	\item Detection (Odhalitelnost)
\end{enumerate}

Pro každou z těchto tří atributů je potřeba stanovit škálu, která nabývá hodnot od jedné do desíti a každé hodnocení obsahuje slovní popis této hodnoty pro jednodušší určení hodnoty. Ukázka možné specifikace bude uvedena při výskytu těchto atributů v následujících částech analýzy.  

Po stanovení těchto atributů je možné přistoupit k vyplnění úvodních informací tvořící hlavičku analýzy. 

\begin{center}
\begin{table}
	\centering
	\caption{Hlavička dokumentu}
	\label{tab:Head1}
        \begin{tabular}{|c | r | c |} 
         \hline
          & Company Name & ABC  \\ [0.5ex] 
         \hline
         1 & Engineering Location & Bratislava  \\ [0.5ex] 
         \hline
         2 & Customer Name & VW Group  \\ [0.5ex] 
         \hline
         3 & Model Year/Program(s) & 2006/J77 \\ [0.5ex] 
         \hline
        \end{tabular}
    \end{table}
\end{center}

Po stanovení těchto atributů je možné přistoupit k vyplnění úvodních informací tvořící hlavičku analýzy. Po stanovení těchto atributů je možné přistoupit k vyplnění úvodních informací tvořící hlavičku analýzy. 
