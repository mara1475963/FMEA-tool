\chapter{Závěr}
\label{sec:zaver}
Cílem této diplomové práce bylo seznámit se s problematikou správy a řízení rizik pomocí metody FMEA a vytvořit nástroj, který bude umožňovat její efektivní vypracování. Součástí práce bylo také seznámení se s dostupnými nástroji a definování požadavků na vytvářený nástroj. Dále měla být definována architektura a návrh jednotlivých částí aplikace. 

Na základě těchto kroků byl vytvořen nástroj pro tvorbu FMEA analýzy, který disponuje několika kvalitami. Důraz by kladen na to, aby tým provádějící analýzu měl možnost pomocí společných setkání tvořit analýzu. Toho bylo docíleno sdílením prováděných změn mezi uživatele připojené do skupiny. Dalším důležitým bodem, na který byl kladen důraz, je přehlednost souvisejících prvků v rámci kroků analýzy. Zde nástroj disponuje přehledným grafickým a tabulkovým režimem pro náhled i tvorbu analýzy. V rámci grafického režimu jsou související atributy sdruženy do objektů stromové struktury. Tyto související atributy lze také přímo označit v tabulce. Pro přehlednost byl také zvolen nový formát tabulky, který se skládá z vertikálaní a horizontální části. Nástroj také podporuje tvorbu uživatelských účtů pomocí, který má přihlášený uživatel další možnosti ve formě ukládání a načítání analýzy jako záznamů v databázi. Součástí je také několik podpůrných funkcí pro tvorbu analýzy jako je tvorba zápisů ze setkání a nastavení vlastních příkladů pro určení  hodnotících atributů. Nakonec je možné exportovat analýzu do několika formátů. Grafická část analýzy je možné uložit do obrázkového souboru a tabulková část do formátu, který podporují tabulkové editory. Je možné i uložení a načtení dat analýzy ze souboru. Cíle práce tedy byly podle autorova názoru dostatečně naplněny a v některých ohledech dokonce možná předčily dostupné komerční nástroje. 

Práce určitě nabízí další možností rozšíření například ve formě nabídky podporovaných typů analýzy, kdy jsou aktualně podporovány nejčastěji použivané typy zaměřené na návrh a proces. Prostor je také v oblasti formátu analýzy, kdy by šlo přidat rozšíření pro jiné oblasti zájmu než automobilový průmysl. Dalším rozšířením by mohlo být lepší možnosti uživatelských účtů. V rámci aplikace by mohly existovat role, na základě kterých by uživatelé měli upravené práva pro modifikaci dat analýzy. Uživatelské účty by také mohly být zakomponovány do přidělování úkolů v rámci kroku pro optimalizaci. Posledním návrhovaným opatřením by mohly být další možností zobrazení výstupních grafů nebo nastavení priority hodnotících atributů, díky kterým by šlo jednodušeju určit, jak naložit s nalezeným rizikem. Tyto rozšíření nebyly impementovány jednak z časových důvodů, ale také kvůli tomu, že částečně přesahují nad rámec zadání této diplomové práce.