% Nejprve uvedeme tridu dokumentu s volbami
\documentclass[czech,master]{diploma}
% Dalsi doplnujici baliky maker
\usepackage[autostyle=true,czech=quotes]{csquotes} % korektni sazba uvozovek, podpora pro balik biblatex
\usepackage[backend=biber, style=iso-numeric, alldates=iso]{biblatex} % bibliografie
\usepackage{dcolumn} % sloupce tabulky s ciselnymi hodnotami
\usepackage{subfig} % makra pro "podobrazky" a "podtabulky"
\usepackage[cpp]{diplomalst}
\usepackage{lscape}

% Zadame pozadovane vstupy pro generovani titulnich stran.
\ThesisAuthor{Marek Bauer}

\ThesisSupervisor{Ing. Jan Kožusznik, Ph.D.}

\CzechThesisTitle{Nástroj pro tvorbu FMEA analýzy}

\EnglishThesisTitle{Tool for FMEA Analysis}

\SubmissionYear{2022}

\ThesisAssignmentFileName{ThesisSpecification_BAU0027.pdf}

% Pokud nechceme nikomu dekovat makro zapoznamkujeme.
\Acknowledgement{Rád bych na tomto místě poděkoval všem, kteří mi s prací pomohli, protože bez nich by tato práce nevznikla.}

\CzechAbstract{}

\CzechKeywords{}

\EnglishAbstract{}

\EnglishKeywords{}

\AddAcronym{DVD}{Digital Versatile Disc}
\AddAcronym{TNT}{Trinitrotoluen}
\AddAcronym{UML}{Unified Modeling Language}
\AddAcronym{HTML}{Hyper Text Markup Language}
\AddAcronym{TUG}{\TeX{} Users Group}

\addbibresource{biblatex-examples.bib}
\addbibresource{coffee.bib}

% Novy druh tabulkoveho sloupce, ve kterem jsou cisla zarovnana podle desetinne carky
\newcolumntype{d}[1]{D{,}{,}{#1}}


% Zacatek dokumentu
\begin{document}

% Nechame vysazet titulni strany.
\MakeTitlePages

% Jsou v praci obrazky? Pokud ano vysazime jejich seznam a odstrankujeme.
% Pokud ne smazeme nasledujici dve makra.
\listoffigures
\clearpage

% Jsou v praci tabulky? Pokud ano vysazime jejich seznam a odstrankujeme.
% Pokud ne smazeme nasledujici dve makra.
\listoftables
\clearpage

% A nasleduje text zaverecne prace.
\chapter{Úvod}
\label{sec:Uvod}
V mnoha odvětvích lidské činnosti je možné přijít do styku s nedokonalostmi v návrzích produktů nebo výrobních procesech, které mohou vést k nezamýšlenému chování a více či méně važným následkům. Následky mohou být triviální, které výrazným způsobem neomezují funkci produktu nebo naopak vážné, které mohou být životu nebezpečné. Dalším důsledkem špatného technologického procesu nebo návrhu je samozřejmě také finanční stránka, kdy nadměrná produkce disfunkčních dílů stojí výrobce peníze navíc za materiál nebo při objevení závady až u koncové zákazníka také výdaje s vyřizováním reklamací, opravami a podobně. Tyto důvody vedly k vytvoření procesů, metod a norem, které mají za cíl odhalení, odstranění nebo alespoň zmírnění možných závad a rizik ještě před začátkem realizace dané činnosti. Souhrně lze tyto snahy označit Risk Management, tedy disciplínu zabývající se správou a řízením rizik. 

Jednou z těchto metod je analytická metoda FMEA(Failure Mode and Effects Analysis), kterou se zabývá tato diplomová práce. Následující kapitola  \ref{sec:FMEA} bude obsahovat základní popis této metody spolu s její historií, dále popisem v jakých oborech se metoda upltňovala nebo aktuálně nejvíce uplatňuje a také rozdělení na základní typy podle toho v jaké fáze vývoje produktu se analýza provádí.

Kapitola \ref{sec:FMEA_postup} se budou zabývat tím, jak se tato analýza provádí. Analýza se skládá celkem ze sedmi kroků, které budou podrobně vysvětleny, budou zde také uvedeny výňatky tabulky z formulářů, ve kterých se analýza provádí. Na jednotlivých krocích analýzy budou zobrazeny odlišnosti dvou typů analýz, na které je kapitola zaměřena. 


\endinput
\chapter{FMEA}
\label{sec:FMEA}
FMEA je zkratka z anglického výrazu Failure Mode and Effect Analysis, česky lze toto označení přeložit jako Analýza možných způsobů a důsledků závad nebo jako Analýza možného výskytu a vlivu vad \cite{preklad}. Jedná se tedy o analytickou metodu, která ma za cíl odhalení možných vad ve výrobním procesu nebo návrhu produktu, nalezení příčin výskytu těchto vad, ohodnocení závažnosti daného rizika a snahu o jeho zmírnění nebo odstranění. Na vypracování této analýzy se většinou podílí tým odborníků z různých oblastí daného odvětví, kteří využívají svých znalostí a zkušeností ze svých profesí k odbornému posouzení problémů v různých fázích analýzy. Obvykle tento tým tvoří vedoucí výroby, zástupce z oddělení pro kontrolu kvality, konstruktér, technolog popřípadě další odborníci. Ideální počet členu by se měl pohybovat okolo 4-6 jedinců. \cite{fmeaTeam} Mezi relevantní účastníky také může patřit zákazník, kterému mohou být poskytovány výstupy z jednotlivých verzí analýzy.  V tomto případě může být FMEA součástí smluvní dohody mezi výrobcem a zákazníkem jako záruka kvality například v rámci PPAP. 

\section{Historie FMEA}
\label{sec:historie}
 Počátky FMEA sahají do 40. let minulého století, kdy byla analýza poprvé použita americkou armádou pro redukci potenciálních závad při výrobě munice. Metoda se ukázala jako vysoce efektivní a kolem roku 1960 ji začlenila do svých přípravných technik i společnost NASA. FMEA se ukázala jako podstatná součást mise Apollo. Od 70.let 20. století již následoval automobilový průmysl, který tvoří jedno z hlavních odvětví, kde je tato metoda využívána. Prvotním uživatelem byla automobilka Ford, která přijala metodu jako reakci na špatný bezpečnostní stav jejich modelu Ford Pinto. Jejich příkladu pak následovali další američtí i evropští výrobci. Tyto události vedly ke vzniku asociací AIAG \cite{aiag} a VDA \cite{vda}, které definují standardy pro zvýšení kvality pomocí nástrojů jako je FMEA, SPC nebo MSA. \cite{historie}

\section{Aplikace FMEA}
 Odpověď na otázku, kdy se FMEA má provádět, se může lišit případ od případu. Nicméně prvotním záměrem je začít provádět metodu, co nejdříve, co jsou k dispozici potřebné informace. Konkrétně se může jednat o případy, kdy jsou vytvořeny nové návrhy,prováděny změny v aktuálních návrzích, změna aplikace stávajících návrhů nebo zlepšení stávajích návrhů. Samozřejmě se může jednat i o celé systémy, procesy, služby atd. Přesto, že je FMEA prováděna před dokončením nebo při dokončení realizace dané fáze vývoje, je dobré přistupovat k dokumentům obsahujícím analýzu jako k dynamickým a v čase zlepšovat jejich kvalitu a přesnost. \cite{fmeatheory} Zaměřením se analýza člení zejména na tři typy, které se odvíjejí podle fáze vývoje. Jedná se o tyto tři druhy: 

\begin{itemize}
	\item  DFMEA (Design)
	\item  PFMEA (Process)
	\item  SFMEA (System)
\end{itemize}

V této diplomové práci bude věnována pozornost hlavně analýze zaměřené na návrh a proces. Tyto dvě zaměření jsou v praxi využívány nejvíce pravděpodobně díky tomu, že se zaměřují na specifičtější části vývoje produktu. Součástí FMEA analýzy je dekompozice produktu ve fázi strukturální analýzy na nízkoúrovňové části, pro které se lépe hledají možné způsoby selhání. Použití metody na celý systém se tak může zdát jako zbytečně komplexní řešení a v některých případech tato možnost ani nepřipadá v úvahu. Konkrétně je dobré uvést příklad z rozsáhlého automobilového průmyslu, kde se výroba vozu skládá ze spolupráce několika různých výrobců komponent, kteří často nemusí mít přehled o práci ostatních dodavatelů. V některých případech jsou výrobky součástí většího celku a těžko se při použití metody DFMEA nalézají všechny možné rizikové scénáře. V těchto případech se například provádí pouze analýza zaměřená na konkrétní výrobní proces, u kterého lze určit možná rizika. 

\subsection{DFMEA}
\label{subsec:DFMEA}
DFMEA se používá pro analýzu nových návrhů produktů. \cite{dfmea} Měla by tedy navazovat na ukončení fáze návrhu a vycházet tak ze softwarových artefaktů z toho vyplývajících. Jedním z těchto artefaktů může být například blokový diagram(Boundary diagram), který určuje rozsah analyzované části systému. Pomocí blokového diagramu lze také zobrazit rozhraní a vztahy mezi jednolivými komponentami. Mezi další podklady pro tvorbu DFMEA mohou patřit: 
\begin{itemize}
	\item  Normy a předpisy
	\item  Fyzikální specifikace materiálů
	\item  Návrhy obdobných produktů, popř. předchozí DFMEA, pokud byla provedena
	\item  Požadavky zákazníka
	\item  Diagram parametrů (P-diagram)
	\item  Shrnutí všech požadavků na návrh
\end{itemize}



\subsection{PFMEA}
\label{subsec:PFMEA}
PFMEA se používá pro analýzu nových výrobních procesů. \cite{pfmea} Dá se říci, že by měla navazovat na předchozí analýzu návrhu produktu, kdy výrobní proces je dalším logickým krokem při vývoji. Mezi vstupní podklady patří například vývojový diagram procesu, který dekomponuje proces do série navazujících kroků. Mezi další podklady pro tvorbu PFMEA mohou patřit: 
\begin{itemize}
	\item  DFMEA
	\item  Specifické zákaznické požadavky
	\item  Zákonné požadavky a předpisy
	\item  Matice znaků
	\item  Zkušenosti z předešlého vývoje podobného procesu
\end{itemize}
V praxi je možné setkat se i s případem, kdy se PFMEA provádí až jako snaha o zlepšení stávajícího procesu. V tomto případě lze jako podklad pro vypracování také vzít v úvahu fyzickou prohlídku výrobní haly. Tuto analýzu často vykonává stejný tým jako v případě DFMEA. Výstupem by měl být tzv. kotrolní plán, který obsahuje seznam konkrétních změn, které by měly být provedeny na základě provedené analýzy. 
\endinput
\chapter{Vypracování FMEA }
\label{sec:FMEA_postup}
V této kapitole bude podrobněji popsán postup při použití metody FMEA pro návrh a proces. Obě tyto varianty mají totožnou strukturu a tak budou její jednotlivé části popsány zároveň pro vytvoření obrazu v jakých atributech se konkrétní prvky obou typů analýz liší. Uvedené příklady budou ze světa automobilového průmyslu. 

Výňatky z formulářů odpovídají standardu příručky FMEA... Jednotlivé reprezentace analýzy se mohou lišit, zejména při použití v odlišných odvětvích. Také je možné se setkat s některými volitelnými atributy, které nemají veliký význam a nebude na ně brán v následujícím popisu zřetel.  

\section{Plánování a příprava(1. krok)}
V kapitolách \ref{subsec:DFMEA} a \ref{subsec:PFMEA}  byly zmíněny předpoklady a vstupní podklady před začátkem samotné analýzy. V tomto kroku je také sestaven tým odborníků, kteří budou za pomocí společných setkání provádět samotnou analýzu. V tomto týmu je potřeba před začátkem práce definovat základní pravidla a pojmy. Je potřeba definovat škálu pro atributy, které pomocí kterých bude probíhat hodnocení nalezených rizik. Od této škály se například bude odvíjet, od jaké hodnoty bude akceptována určitá míra rizika. Tyto atributy jsou: 

\begin{enumerate}
	\item Severity (Vážnost)
	\item Occurance (Výskyt)
	\item Detection (Odhalitelnost)
\end{enumerate}

Pro každou z těchto tří atributů je potřeba stanovit škálu, která nabývá hodnot od jedné do desíti a každé hodnocení obsahuje slovní popis této hodnoty pro jednodušší určení hodnoty. Ukázka možné specifikace bude uvedena při výskytu těchto atributů v následujících částech analýzy.  

Po stanovení těchto atributů je možné přistoupit k vyplnění úvodních informací tvořící hlavičku analýzy. 

\begin{center}
\begin{table}
	\centering
	\caption{Hlavička analýzy}
 \label{tab:header_FMEA}
	\label{tab:Head1}
        \begin{tabular}{|c | r | c |} 
         \hline
 \multicolumn{3}{|c|}{Planning and preparation (Step 1)} \\

         \hline
         1 & Company Name & ABC  \\ [0.5ex] 
         \hline
         2 & Location & Bratislava  \\ [0.5ex] 
         \hline
         3 & Customer Name & VW Group  \\ [0.5ex] 
         \hline
         4 & Model Year / Program(s) & 2006/J77 \\ [0.5ex] 
         \hline
         5 & Subject & J77 Instrument Cluster \\ [0.5ex] 
         \hline
         6 & Responsibility & John Doe (Product Eng.) \\ [0.5ex] 
         \hline
         7 & D(P)FMEA Start Date & 1.1.2023  \\ [0.5ex] 
         \hline
         8 & D(P)FMEA Revision Date & 27.2.2023 \\ [0.5ex] 
         \hline
         9 & D(P)FMEA ID Number & 77 \\ [0.5ex] 
         \hline
         10 & Confidentiality Level & Proprietary \\ [0.5ex] 
         \hline
         11 & Cross-Functional team & John Doe, Henry Smith \\ [0.5ex] 
         \hline
        \end{tabular}
    \end{table}
\end{center}

Tato tabulka \ref{tab:header_FMEA} obsahuje seznam atributů, které se vyplňují v rámci úvodní fáze analýzy. Zde je popis jednotlivých atributů:

\begin{enumerate}
	\item \textbf{Název společnosti} - jedná se o společnosti v rámci, které je analýza prováděna
	\item \textbf{Lokace} - zde může lokace odpovídat buď lokací inženýrského týmu zodpovědného za návrh produktu(DFMEA) nebo umístěním výrobního závodu(PFMEA) 
	\item \textbf{Jméno zákazníka} - Podle zaměření analýzy, může být zákazníkem jak koncový uživatel, tak i interní oddělení, které navazuje ve výrobním cyklu
 \item \textbf{Modelový rok, Program} - specifikace modelu produktu, program(DFMEA) nebo platforma(PFMEA)
 \item \textbf{Předmět} - vysokoúroňový popis zkoumaného prvku
 \item \textbf{Zodpovědnost} - vlastník dokumentu analýzy, často v roli manažera setkání týmu
 \item \textbf{Začátek analýzy} - počáteční datum
 \item \textbf{Datum revize} 
 \item \textbf{Interní identifikační číslo analýzy}
 \item \textbf{Stupeň důvěrnosti} - může nabývat hodnot (obchodní, proprietární, důvěrný)
 \item \textbf{Multioborový tým} - seznam lidi podílejících se na tvorbě analýzy
\end{enumerate}

\section{Analýza struktury(2. krok)}
\label{sec:FMEA_postup_2}
Cílem strukturální analýzy je dekompozice zkoumaného produktu na menší části, které pak slouží jako vstupní parametry v dalších krocích analýzy. Dekompozice slouží pro jednodušší pochopení zkoumaného produktu. Výsledkem je nejčastěji stromová struktura o výšce stromu tři, kdy kořenem stromu je přímo zkoumaný prvek.





\begin{center}
\begin{table}[h]
	\centering
	\caption{Formulář pro analýzu struktury(DFMEA) }
	\label{tab:structure_DFMEA}
\begin{tabular}{ |p{4cm}|p{3cm}|p{3cm}|  }
 \hline
 \multicolumn{3}{|c|}{Structure Analysis (Step 2)} \\
 \hline
 1. Next Higher Level& 2.Focus Element
No. And Name of Focus Element &3. Next Lower Level of Characteristic Type\\
 \hline
 Instrument Cluster B90   & 6. Speed clock    &Stepper motor\\


 \hline
\end{tabular}\  
\end{table}
\end{center}

\begin{center}
\begin{table}[h]
	\centering
	\caption{Formulář pro analýzu struktury(PFMEA) }
	\label{tab:structure_PFMEA}
\begin{tabular}{ |p{4cm}|p{3cm}|p{3cm}|  }
 \hline
 \multicolumn{3}{|c|}{Structure Analysis (Step 2)} \\
 \hline
1. Process Item
System, Subsystem, Part Element or Name of Process
& 2. Process Step
No. And Name of Focus Element
&3. Process Work Element

4M Type\\
 \hline
X98 sun roll assembly line   & 020 Assembly body-mechanism   &Operator\\


 \hline
\end{tabular}\  
\end{table}
\end{center}

V tomto kroku analýzy se nejvíce projevují rozdíly mezi jejími dvěma zmíněnými typy. Rozdílnost atributů, které jsou v této fázi vyplňovány je možné porovnat v tabulkách \ref{tab:structure_DFMEA} a  \ref{tab:structure_PFMEA} 
Dále následuje popis jednotlivých atributů, které se vyplňují v rámci strukturální analýzy. 

\begin{enumerate}
	\item \textbf{Vyšší úroveň / Položka procesu systému, subsystému, dílu nebo název procesu} - Prvek na nejvýšší úrovni v rámci předmětu analýzy, pro tento prvek se v se v rámci analýzy selhání definují důsledky selhání
	\item \textbf{Vybraný prvek, číslo a označení/Krok procesu, číslo a označení} - Prvek na druhé úrovni, pro tento prvek se v rámci analýzy selhání určuje jeden ze základních pílířů analýzy a to způsob selhání
	\item \textbf{Nižší úroveň nebo druh charakteristiky/Prvek provádění činnosti, 4M} - Prvek na nejnižší úrovni, pro který se určují příčiny selhání. V analýze zaměřené na proces je zde uveden návodně typ 4M, který vyjadřuje(Machine, Man, Material, Milieu), tedy určité kategorie, do kterých by mohl hodnotící tým zařadit prvek na této úrovni. Další kategorie mohou také být Method, Measurement.  
\end{enumerate}

\section{Analýza funkcí(3. krok)}
\label{sec:FMEA_postup_3}
V rámci analýzy struktury byly vytvořeny objekty, pro které ve fázi analýza funkcí bude tým definovat jejich funkce. Nalezené funkce budou dále sloužit jako vstupní parametry pro analýzu selhání. 

Stejně jako v předchozím kroku, kdy výsledná analýza tvořila stromovou strukturu, neboli jednotlivé objekty měli mezi sebou definované vazby, tak je tomu obdobně i v této části analýzy. Zde je kladen důraz hlavně na funkcionalitu prvků na druhé úrovni. Pro tyto funkce se pak hledají funkce ze třetí úrovně, které zajišťují jejich funkcionalitu. Stejně je potřeba nalézt funkce na první úrovni, které jsou chápany jako výsledky funkce ze druhé úrovně. 

Analýza funkcí je dalším příkladem, kdy se jednotlivé typy analýzy zaměřené na návrh a proces liší. Tyto odlišnosti znázorňují tabulky \ref{tab:function_DFMEA} a \ref{tab:function_PFMEA} 

\begin{center}
\begin{table}[h]
	\centering
	\caption{Formulář pro analýzu funkcí(DFMEA) }
	\label{tab:function_DFMEA}
\begin{tabular}{ |p{5cm}|p{4cm}|p{4cm}|  }
 \hline
 \multicolumn{3}{|c|}{Function Analysis (Step 3)} \\
 \hline
 1. Next Higher Level Function and Requirement &
2. Focus Element
Function and Requirement &
3. Next Lower Level Function and Requirement or Characteristic\\
 \hline
 Display vehicle status: spped, torq, gas level, engine temperature..   & Display vehicle speed    & Convert a train of input pulses into a precisely defined increment in the shaft position\\


 \hline
\end{tabular}\  
\end{table}
\end{center}

\begin{center}
\begin{table}[h]
	\centering
	\caption{Formulář pro analýzu funkcí(PFMEA) }
	\label{tab:function_PFMEA}
\begin{tabular}{ |p{5cm}|p{4cm}|p{4cm}|  }
 \hline
 \multicolumn{3}{|c|}{Function Analysis (Step 3)} \\
 \hline
1. Function of the Process Item
Function of System, Subsystem, Part Element or Name of Process
& 2. Function of the Process Step and Product Characteristics
(Quantitative value is optional)
& 3. Function of the Process Work Element and Process Characteristics
\\
 \hline
Assembly body with mechanism   & Screw body with mechanism with 2 screws   &Take the first screw and insert into driver then press tool start button\\


 \hline
\end{tabular}\  
\end{table}
\end{center}

Popis atributů v rámci analýzy funkcí:
\begin{enumerate}
	\item \textbf{Funkce a požadavek vyšší úrovně / Funkce položky procesu Funkce systému, subsystému, dílu/komponentu nebo procesu} - Funkce odpovídající prvku na nejvyšší úrovni nebo procesu 
	\item \textbf{Funkce a požadavek vybraného prvku/Funkce kroku procesu a charakteristika produktu(kvantitaivní hodnota je volitelná)} - Funkce odpovídající prvku na druhé úrovni(subsystém, komponent) nebo kroku procesu(čeho musí daná stanice dosáhnout)
	\item \textbf{Funkce a požadavek nebo charakteristika nižší úrovně/Funkce prvku provádějící činnost a charakteristika procesu} - Funkce odpovídající prvku na nejnižší úrovni(komponent, díl) nebo prvku provádějící nějakou činnost v procesu
\end{enumerate}



\section{Analýza selhání(4. krok)}
Analýza selhání je klíčovým krokem v rámci celé anýlýzy. Z předchozí činnosti, kdy byly objektům přiřazeny funkce, je cílem tohoto kroku zjistit, jak by objekt o dané funkci mohl selhat. Pro nalezené selhání se dále hledají důsledky a příčiny, které jsou přiřazeny funkcím na první a druhé úrovni v rámci celkové struktury. 

Stejně jako u předchozích dvou kroků popsaných v kapitolách \ref{sec:FMEA_postup_2} a \ref{sec:FMEA_postup_3} existují mezi atributy relační vazby. Zde je celkem logicky každému selhání přiřazeno několik možných důsledků a příčin. Důsledky selhání jsou definovány jako dopad na celkový systém, a proto náleží kořenovému objektu. Příčiny selhání jsou definovány jako selhání prvků na nejnižší úrovni dekompozice a slouží spolu s ostatními atributy jako vstupní parametry pro analýzu rizik v následujícím kroku.

Analýza selhání je krok, při kterém se již dva zmiňované typy od sebe tolik neliší. Význam atributů je prakticky stejný, akorát se odkazují na různé objekty z analýzy struktury. Porovnání obou typů zobrazují tabulky \ref{tab:failure_DFMEA} a \ref{tab:failure_PFMEA}


\begin{center}
\begin{table}[htp]
	\centering
	\caption{Formulář pro analýzu selhání(DFMEA) }
	\label{tab:failure_DFMEA}
\begin{tabular}{ |p{4cm}|p{0.5cm}|p{3cm}|p{4cm}|  }
 \hline
 \multicolumn{4}{|c|}{Failure Analysis (Step 4)} \\
 \hline
  1. Failure Effects (FE) to the next higher level and/or End User&
  \begin{turn}{-90}Severity (S) of FE\end{turn} &
2. Failure Mode (FM) of the Focus Element &
3. Failure Cause (FC) of the Next Lower Level Element or Characteristic
\\
 \hline
 Speed not display.
  & 5
  & Speedometer needle lock to 0 position. & Lost connection between stepper motor and board due bad soldering.
\\
 \hline
\end{tabular}\  
\\
\hfill \break
\hfill \break
\centering
	\caption{Formulář pro analýzu selhání(PFMEA) }
	\label{tab:failure_PFMEA}
\begin{tabular}{ |p{4cm}|p{0.5cm}|p{3cm}|p{4cm}|  }
 \hline
 \multicolumn{4}{|c|}{Failure Analysis (Step 4)} \\
 \hline
  1. Failure Effects (FE)
&
  \begin{turn}{-90}Severity (S) of FE\end{turn} &
2. Failure Mode (FM) of the Process Step
 &
3. Failure Cause (FC) of the Work Element

\\
 \hline
Plant:
Gap between housing and mechanism more than tolerance - rework necessary (less than 100% of batch)
Ship to Plant:
Additional insertion force to assembly it into door panel
End user:
Possibility in time to hear noises (disturbed)

  & 5
  & Missing one screw
 & Operator skip

\\
 \hline
\end{tabular}\  
\end{table}
\end{center}



\clearpage
Popis atributů v rámci analýzy selhání:
\begin{enumerate}
	\item \textbf{Následek (FE) na prvku vyšší úrovně/koncového uživatele } - Důsledek selhání odpovídající prvku na nejvyšší úrovni. Jak je vidět na uvedeném příkladu pro PFMEA, tak je možné držet se určitých mantinelů pro definování dopadů selhání. Konkrétně je možné určit dopady na vlastní továrnu, továrnu zákazníka nebo koncového uživatele. 
 
 Ve formulářích pro analýzu selhání je uveden i atribut Závažnost, který se ale hodnotí až v rámci analýzy rizik. Proto mu bude věnováno více pozornosti až v kapitole \ref{sec:FMEA_postup_5}
	\item \textbf{Vada(FM) vybraného prvku/ Vada(FM) kroku procesu} - Selhání(vada) prvku nebo procesního kroku na druhé úrovni. Stanovení hodnocení tohoto atributu je důležité i z hlediska názvosloví a je potřeba používat odbornou a ucelenou terminologii.  
	\item \textbf{Přičina(FC) na prvku nižší úrovně nebo charakteristiky/Přičina(FC) v prvku provádějící činnost} - Přičina selhání odpovídající prvku na nejnižší úrovni. V případě procesní analýzy odpovídá prvku z kategorie 4M. 
\end{enumerate}




\section{Analýza rizik(5. krok)}
\label{sec:FMEA_postup_5}
Analýza rizik vychází z předchozího kroku, kdy hodnotící tým stanovil možné způsoby selhání a jejich příčiny. V rámci analýzy tohoto kroku bude tým odborníků hodnotit toto riziko na základě aktuálně stanovených opatření. Výstupem hodnocení pro každé riziko musí být jedna z následujích variant:
\begin{itemize}
	\item Aktuální opatření jsou dostačující/hodnocení rizika není na tolik závažné, aby bylo nutné přistupovat k nějakým změnám. 
    \item Riziko může mít negativní dopady do té míry, že je potřeba závést nové opatření nebo upravit ty stávající tak, aby došlo ke zmírnění hodnocení některých atributů.
\end{itemize}
Analýza jako celek nemůže být uzavřena, dokud nejsou všechny rizika v jednom z uvedených stavů.

Tato fáze analýzy je pro oba typy FMEA prakticky totožná a význam atributů je stejný, proto bude uveden pouze jeden společný příklad.

\begin{center}
\begin{table}[h]
	\centering
	\caption{Formulář pro analýzu rizik }
	\label{tab:risk_FMEA}
\begin{tabular}{|p{4cm}|p{0.5cm}|p{4cm}|p{0.5cm}|p{0.5cm}|  }
 \hline
 \multicolumn{5}{|c|}{Risk Analysis (Step 5)} \\
 \hline
1. Current preventive control (PC) for FC
&
  \begin{turn}{-90}2. Occurence (O) of FC\end{turn} &
3. Current detection control (DC) for FC or FM
 &
  \begin{turn}{-90}4. Detection (D) of FC or FM\end{turn}
 &
  \begin{turn}{-90}5. D(P)FMEA AP\end{turn}

\\
 \hline
No prevention.
& 10
& Testing method to be developed.
& 10
& H


\\
 \hline
\end{tabular}\  
\end{table}
\end{center}


Popis atributů v rámci analýzy rizik:
\begin{enumerate}
	\item \textbf{Stávající preventivní opatření k příčině} - Příklad opatření, které firma aktuálně využívá a slouží jako prevence selhání. 
	\item \textbf{Výskyt příčiny} - Jeden ze tří hodnotících atributů. Nabývá hodnot nejčastěji ze stupnice 1-10. Číselnému hodnocení odpovídá také hodnocení slovní, které usnadňuje stanovení číselné hodnoty. Příklad tabulky, která se používá pro hodnocení výskytu selhání je součástí přílohy... Pro oba typy FMEA se používá odlišná tabulka. Jeden z aspektů pro hodnocení je také předchozí atribut. Dá se říci, že čím kvalitnější jsou stávající preventivní opatření, tím bude i hodnocení tohoto atributu menší.     
	\item \textbf{Stávající opatření pro detekci příčiny nebo selhání} - Metoda, kterou tým v rámci tohoto atributu definuje již není preventivní, ale slouží k odhalení už vzniklého selhání. 
	\item \textbf{Detekce příčiny} - Poslední hodnotící atribut. Stejně jako u atributu výskytu, tak dosahuje hodnot 1-10, které odpovídá i slovní popis. Taktéž navazuje na předchozí atribut, který také hraje roli v tom jak moc vysoká bude číselná hodnota detekce. 
	\item \textbf{Priorita (Action Priority) v rámci DFMEA nebo PFMEA} - 
 Ještě před popisem posledního atributu je potřeba zmínit atribut Závažnost, který byl součástí předchozí ukázky formuláře v tabulce \ref{tab:failure_DFMEA} nebo \ref{tab:failure_PFMEA}. Závažnost je první z hodnotících atributů, který má obvykle v rámci celkového hodnocení rizika největší váhu. Závažnost se váže na množinu dopadů odpovídající nějakému selhání tzn. že pro více hodnot Výskyt a Detekce může být použita společná hodnota atributu Závažnost. Hodnocení tohoto atributu probíhá stejným způsobem jako u Výskytu a Detekce, nicméně tento atribut vyjadřuje i určité dopady selhání nebo příčinny na koncového zákazníka. Tyto dopady mohou být triviální, kdy dochází například pouze k ovlivnění vzhledu, zvuku nebo vibracím, ale také velice závažným, kdy může být uživatel ohrožen na životě. Příkladem takové situace je, když je analyzovaným produktem nějaký důležitý komponent automobilu. Tabulka sloužící pro stanovení hodnoty je taktéž součástí příloh....
 
 Poslední atribut analýzy rizik slouží jako součást rozhodnutí, jak s odhaleným rizikem naložit. Tento atribut vychází ještě z jedné hodnoty, která se nicméně již v nových verzích analýzy neuvádí. Touto hodnotou je RPN(Risk Priority Number) a vyjadřuje jednoduše součin číselného hodnocení atributů Závažnost, Výskyt a Detekce. Logicky se bude tato hodnota pohybovat v rozmezí 1-1000. O výpočet této hodnoty se z pravidla stará použitý software. Na základě vypočtené hodnty RPN je stanovena i hodnota atributu AP, která může nabývat hodnot(High, Medium, Low) .

Jak již bylo řečeno AP je pouze součástí pro rozhodnutí do jakého stavu dané riziko přejde. Tím, že AP vychází ze součinu, jehož tři činitelé mají stejnou váhu, tak nemusí hodnota plně reflektovat stav daného rizika. Pro ucelené a konečné hodnocení je potřeba v rámci týmu vzít v úvahu o jakou konkrétni situaci se jedná a jak moc mají stanovené hodnoty v tomto případě váhu. 
 
\end{enumerate} 
\section{Optimalizace(6. krok)}
\label{sec:FMEA_postup_6}
Ná základě předchozího kroku tým vykonávající analýzu vyhodnotil pro jaké rizika je potřeba zavést dodatečné opatření pro zmírnění některého z atributů a snížení celkového hodnocení v rámci AP. Těchto rizik se bude týkat předposlední krok a to je Optimalizace. Zjednodušeně je cílem tohoto kroku stanovit nové opatření pro prevenci a detekci selhání, přiřadit provedení změn konkrétní osobě, určit datum revize a po uplnutí tohoto data provést opětovné hodnocení atributů SOD s předpokladem, že bude výsledná hodnota rizika zmírněna.

Tento krok je posledním v rámci formuláře, který má skupina hodnotitelů k dispozici. Stejně jako tomu bylo u několika předchozích kroků, tak je význam atributů v tomto kroku společný pro oba typy analýzy. Dále následuje tabulka \ref{tab:optimization_FMEA}, která obsahuje seznam všech atributů, které se v rámci tohoto kroku vyplňují.

\begin{landscape}
\begin{table}[h]
	\centering
	\caption{Formulář pro Optimalizaci }
	\label{tab:optimization_FMEA}
\begin{tabular}{|p{2.5cm}|p{2.5cm}|p{2.5cm}|p{2.5cm}|p{1.5cm}|p{2.5cm}|p{2.5cm}|p{0.5cm}|p{0.5cm}|p{0.5cm}|p{0.5cm}| }
 \hline
 \multicolumn{11}{|c|}{Optimization (Step 6)} \\
 \hline
1. D(P)FMEA Prevention Action &
2. D(P)FMEA Detection Action &
3. Responsible Persons Name &
4. Target Completion Date &
5. Status &
6. Action taken with Pointer to Evidence &
7. Completion Date &

\begin{turn}{-90}8. Severity(S)\end{turn} &
\begin{turn}{-90}9. Occurence (O)\end{turn} &
\begin{turn}{-90}10. Detection (D)\end{turn} &
\begin{turn}{-90}11. D(P)FMEA AP\end{turn}

\\
 \hline
Re-use same design as for X44 project - proved solution. &
Update test method - test to failure method.& 
John Doe & 
15.05.2020 &
C & 
Design Review 244243 May 27 2020 & 
15.05.2020 & 
5 & 
3 & 
3 & 
L 

\\
 \hline
\end{tabular}\  
\end{table}
\end{landscape}

Popis atributů v rámci optimalizace:
\begin{enumerate}
	\item \textbf{Preventivní opatření} - Zavedení nového preventivního opatření nebo úprava stávajícího
	\item \textbf{Opatření k odhalení} - Zavedení nového opatření pro detekci chyby nebo úprava stávajícího
	\item \textbf{Odpovědná osoba} - Přiřazení realizace změn v rámci opatření konkrétní osobě. Tato osoba by měla být součástí týmu a také uvedena ve skupině účastníků v hlavičce dokumentu.
	\item \textbf{Plánované datum dokončení}
	\item \textbf{Status} - Tento atribut udává v jakém stavu je realizace změn v opatřeních. Může nabývat například hodnot (O - Open, DP - Decision Pending, IP - Implementation Pending, C - Completed, NP - Not Implemented)
	\item \textbf{Přijatá opatření s odkazem na důkaz} - Dokumentace provedených změn. Měl by obsahovat i odkaz na další dokument popisující provedené opatření. 
	\item \textbf{Datum dokončení} 
	\item \textbf{Význam} - Opětovné hodnocení významu selhání
	\item \textbf{Výskyt} - Opětovné hodnocení výskytu selhání
	\item \textbf{Detekce} - Opětovné hodnocení detekce selhání
	\item \textbf{Priorita} - Po stanovení všech nových opatření se provádí reevaluace všech hodnotících atributů. Cílem je zjistit, jestli byla provedená opatření natolik účinná, že je možné ohodnotit jednotlivé atributy SOD nižší známkou a tím také snížit prioritu rizika. 
\end{enumerate}

\section{Dokumentace výsledků(7. krok)}
Dokumentace výsledků je posledním krokem v rámci vypracování FMEA. Jeho účelem je sumarizace dosažených výsledků a provedených zlepšení v rámci návrhu produktu nebo výrobního procesu. Jak již bylo řečeno předpokladem pro provedení dokumentace je to, že byly uzavřeny všechny nalezené rizika. Dokumentace výsledků již není součástí vyplňovaného formuláře. Dokument může mít libovolný formát vyhovující potřebám firmy. FMEA může sloužit i jako záruka kvality při komunikaci se zákazníkem v roli obchodního řetězce prodávájící daný produkt. Pro tyto účely je možné také použít shrnutí vytvořené v rámci tohto kroku. Shrnutí také může sloužit pro informování vysoko postavených osob ve vedení firmy. 
  
\chapter{Analýza a specifikace požadavků}
\label{sec:pozadavky}
Tato kapitola bude zaměřena na popis požadavků na vyvýjený nástroj. Tento popis bude inspirován disciplínou inženýrství požadavků, která je využívána v oboru softwarového inženýrství pro analýzu, specifikaci a dokumentaci požadavků. Tato disciplína je využívána zejména ve smyslu komunikace se zákazníkem, kdy na základě společných jednání, je postupně tvořena představa o výsledném produktu. Tato představa není nikdy ve fázi návrhu kompletní a proto se mohou jednotlivé požadavky v průběhu vývoje měnit. Vzniklé artefakty mohou být například:
    \begin{itemize}
    \item Vize
	\item Specifikace zákaznických požadavků
    \item Specifikace systémových/softwarových požadavků
    \item AND/OR graf
    \item I* graf
    \item KAOS graf 
\end{itemize}

V rámci této diplomové práce neprobíhala komunikace se zákazníkem, nicméně se specifikace požadavků na základě praktit používaných v inženýrství požadavků ukázala být přínosem. Požadavky vycházely zejména ze zadání diplomové práce, konzultace s vedoucím práce a rešerše existujích řešení popsané v kapitole \ref{sec:nastroje}. Výsledkem je kombinace specifikací zákaznických a systémových požadavků. Seznam požadavků na výsledný nástroj pro tvorbu FMEA je dostupný v tabulce \ref{tab:pozadavky}
\break
\break
\break
\break
\break




        \begin{longtable}{|p{0.5cm} | p{12cm} | p{0.5cm} | p{0.75cm} | p{0.5cm} |} 
        \caption{Specifikace požadavků}
\label{tab:pozadavky}
         \hline
         ID & 
         Popis & 
         \begin{turn}{-90}Typ záznamu\end{turn} & 
         \begin{turn}{-90}Druh požadavku \end{turn}& 
         \begin{turn}{-90}MoSCoW \end{turn} \\ \hline
 1 &	\textbf{Tvorba FMEA}  &	H & & \\
 2 &	\textit{Základní požadavek na vyvýjený software je efektivní způsob tvorby FMEA.} &	I	&& \\			
3 &	Systém bude podporovat tvorbu analýzy zaměřené na návrh a proces(Design, Process) & R & CR & M \\
4 &	Systém bude podporovat náhled a tvorbu analýzy pomocí textové tabulky. & 	R &	CR & M\\
5 &	Systém bude podporovat náhled a tvorbu analýzy v grafickému režimu. &	R &	CR & M\\
6&	Systém bude automaticky počítat hodnoty RPN a určení AP.	&R&	CR&M\\
7&	Systém bude umožňovat řazení rizik podle AP. &	R &	CR & W\\
8 &	Systém bude zobrazovat relace mezi atributy v rámci kroků analýzy 1,2,3. &	R &	CR & S\\
9&	Data bude možné editovat pomocí modálních oken. &	R	&SYS & S\\
10 &	Tvorba analýzy v grafickém režimu bude realizována drag\&drop stylem. &	R &	SYS&W\\
11 &	\textbf{Vstupy a výstupy projektu}  &	H & & \\
12 &	\textit{Podpůrné funkce nástroje pro vizualizaci výsledků, nastavení projektu apod.} &	I	&& \\
13	&Systém bude nabízet vychozí slovní definici pro hodnotící parametry(Význam, Výskyt, Detekce) dle standardu AIAG/VDA pro automotiv. & 	R &	CR & M \\
14&	Systém bude podporovat definici vlastních měřítek pro určení hodnotících parametrů.&	R&	CR&S\\
15&	Systém bude umožňovat vytvářet logy z jednotlivých setkání týmu. &	R&	CR&S\\
16 & Systém bude umožňovat zobrazení shrnutí rizik spolus s grafovou reprezentací pro kontrolu úplnosti provedené analýzy.&	R&	CR&M\\
17 & Graf rizik bude sloupcového typu a bude obsahovat počáteční a konečné hodnoty AP. &	R&	SYS. &S\\
 18 &	\textbf{Uživatelé}  &	H & & \\
19 &	\textit{Analýza je prováděna v rámci týmu odborníku, kterým nástroj usnadňuje podmínky při tvorbě anlýzy.
} &	I	&& \\
20 &	Systém bude podporovat použití jedním či více uživateli zároveň.&	R&	CR&M\\
21&	Systém bude podporovat registraci (vytváření uživatelských učtů)	&R	&CR&S\\
22&	Systém bude umožňovat přihlášení k uživatelskému účtu.&	R&	CR&M\\
23&		Systém bude sdílet prováděné změny ve skupině připojených uživatelů v reálném čase. &	R&	CR&S\\
24&		Systém bude umožňovat sdílení pomocí url odkazu.&	R&	SYS&M\\
24&		Pro přihlašování bude možné použít email a heslo nebo Google účet. &	R&	SYS&M\\


25 &	\textbf{Data}  &	H & & \\
26 &	\textit{Tato kapitola definuje požadavky na nástroj z pohledu persistence dat, jejich formátu apod.} &	I	&& \\
27 &	Systém bude umožňovat uživateli ukládání dat analýzy. &	R&	CR&M\\
28 &	Systém bude umožňovat uživateli načítání dat vlastních analýz. &	R&	CR&M\\
29 &	Systém bude umožňovat export analýzy do formatů .xls, .json, .png &	R &	CR & S\\
29 &	Systém bude umožňovat import analýzy z .json formátu &	R &	CR & M\\
30 &	Systém bude provádět kontrolu správnosti formátu importovaných dat. & 	R &	SYS & S\\
31 &	Analyza bude ukladána jako záznamy do relační databáze.& 	R&	SYS&W\\
32&	\textbf{Ostatní}  &	H & & \\
33 &	\textit{Další obecné požadavky na vyvýjený systém, včetně tzv. nefunkčních požadavků.} &	I	&& \\
34	& Nástroj bude dostupný na zařizeních nezávisle na OS. &	R&	CR& M\\
35	& Systém bude zvládat současný přistup několika desítek uživatelů. &	R&	CR& S\\
36	& Systém bude podporovat zobrazení na mobilních zařízeních	&R	&CR&W\\ \hline
\end{longtable}

\section{Popis tabulky}
Tabulka je členěna do kapitol, kterým odpovídá určitá sada záznamů. Pro toto členění je použit sloupec Typ záznamu, který může nabývat hodnot:
    \begin{itemize}
    \item H(header) - název kapitoly
	\item I(Information) - úvodní informace o kapitole
    \item R(requirement) - požadavek
\end{itemize}
Každý záznam také obsahuje svoje identifikační číslo, které může sloužit jako odkaz v jiných dokumentech. Požadavky jsou rozlišeny také na základě jejich druhu, a to na zákaznické a systémové. Zákaznické požadavky vyjadřují zejména nějakou funkci systému z pohledu uživatele. Zdrojem těchto požadavků je zákazník. Systémové požadavky vyjadřují spíše, jak bude určitá funkce implementována. Může se jednat o více či méně technický popis použitého postupu pro realizaci požadavku. Zdrojem požadavků je vývojář, programátor,... Poslední sloupec tabulky vyjadřuje prioritu požadavku. Může nabývat hodnot:
    \begin{itemize}
    \item M(Must)
	\item S(Should)
    \item  C(Could)
    \item  W(Wont)
    
\end{itemize}
Slouží například k určení jaké požadavky budou implementovány v prvé řadě, a které budou doplńovány postupně podle zbylého času a dalších prostředků. Požadavky vyplývající přímo ze zadání diplomové jsou označení jako primární. Požadavky vzniklé  inspirací z existujích řešení jsou označeny písmenem S - nástroj by měl obsahovat tuto funkci. Požadavky ozačené písmenem C jsou takové, které mohou a nemusí být implementovány. O tom bude rozhodnuto v průběhu implementace. Poslední skupina jsou zamítnuté požadavky. Ty se v průběhu vývoje ukázaly jako neužitečné. Dále také mohou přímo určovat, co systém nebude umožňovat. Slovní formát samotných požadavků má cíleně jednou strukturu.  

\section{Popis vybraných požadavků}
V této kapitole budou detailněji popsány některé požadavky. Popis bude obsahovat důvody, proč byly jednotlivé požadavky přijmuty nebo zamítnuty. Odkazem na požadavek bude identifikační číslo z tabulky. 
\subsection{Požadavek 3}
Jedná se o základní požadavek, který vycházel jak z odborné literatury, tak z rešerše dostupných řešení na trhu. V praxi se nejvíce používá FMEA v automobilovém průmyslu a existující nástroje z velké části podporují tvorbu analýzy zaměřené na Návrh a Proces. Samozřejmě existují i analýzy zaměřené například na Systém nebo nově tzv. FMEA-MSR(Monitoring and System response), která byla publikována v posledním vydaní příručky v rámci AIAG/VDA[1]. .. Nicméně se jedná spíše o doplňkové FMEA a proto na ně nebyl kladen důraz. Požadavek byl konzultován a schável vedoucím práce a přijat v aktuálním znění.  

\subsection{Požadavek 7}
Tento požadavek byl vytvořen v rané fázi vývoje jako snaha o zjednodušení určení největších rizik tím, že bude nástroj podporovat jejich řazení podle hodnoty atributu AP(od nejzávažnějších po mírné). Nicméně v průběhu implementace se ukázalo, že samotná hodnota AP nemá tak velikou vypovídající hodnotu z toho důvodu, že se jedná víceméně o jinou reprezentaci RPN, což je součin tří hodnotích atributů. Při rozhodnutí o tom jaká akce bude s rizikem provedena je potřeba vzít v úvahu konkrétní situaci a váhu jednotlivých atributů, ne pouze hodnotu AP. Pro tým by tedy toto řazení mohlo být zavádějící a z toho důvodu byl požadavek zamítnut.   

\subsection{Požadavek 10}
\label{subsec:pozadavek_10}
Dalším zamítnutým požadavkem je návrh toho, jak by mohl být vyřešen požadavek na náhled a tvorbu analýzy v grafickém režimu. Jedná se o systémový požadavek. V jednom ze zkoumaných nástrojů byl použit právě styl drag\&drop, který vypadal jako relativě atraktivní způsob v rámci strukturální analýzy. Nicméně při hledání možnosti, jak zobrazit v aplikaci stromovou strukturu dat, tak byla nalezena knihovna, která tuto možnost nepodporovala. V jiných ohledech tak přesně odpovídala požadovným vlastnostem a proto byl nakonec zvolen styl přidávání dalších uzlů pomocí tlačítka pro přidání. 

\subsection{Požadavek 23}
Při hledání vize projektu se nabízely dvě možnosti, jak realizovat společnou práci uživatelů při použití nástroje. První možností byla řekněme podniková aplikace, která by nabízela uživatelům možnost tvorby FMEA analýzy spolu se správou uživatelů a sdílení dokumentu analýzy skupině uživatelů na úrovní databáze. Druhá, mnohem zajímavější možnost, byla vytvořit aplikaci, kdy se skupina uživatelů připojí přes jeden odkaz a bude mít možnost společně tvořit analýzu pomocí sdílení prováděných změn v reálném čase. Tato možnost mnohem více reflektovala vizi toho, jak by se analýza měla ideálně provádět a proto byl přijat i tento požadavek.  

\subsection{Požadavek 31}
\label{sec:pozadavekMongo}
Tento systémový požadavek vyjadřuje prvotní myšlenku toho, jak by měly být data ukládány ve smyslu databáze a persistentního uložení dat. Vzhledem k tomu, že z hlediska teoria by měly existovat relační vazby mezi jednotlivými entitami v rámci strukturální, funkční a analýzy selhání, tak se jevila jako správna možnost uložení dat relační databáze. Ve fázi návrhu a rané implementace se ukazovala i jako druhá alternativa použití tzv. No-SQL databáze. Což je odlišný přístup k uložení dat, kdy jsou data ukládany v rámci dokumentů, které mohou mít mnohem flexibilnější možnosti např. z hlediska podporovaných datových typů. Jak označení napovídá, tak pro dotazy nad daty není použit jazyk SQL, ale používá se speciálních metod s definovánými parametry pro selekci, vkládání, aktualizování a mazání záznamů. Autor této práce měl zájem na tom vyzkoušet tuto alternativu, která se čím dál tím více začíná používat v moderních řešení webových aplikací. Dalším důvodem proč byla nakonec zvolena No-SQL databáze byla také knihovna sloužící pro zobrazování stromové struktury v rámci popisu požadavku v kapitole \ref{subsec:pozadavek_10}. Komponenta této knihovny tvořící stromovou strukturu dat, přijímá jako vstupní parametr pouze objekt o definovaných a volitelných atributech. V případě této apliakce se jedná vlastně o několikanásobně zanořený objekt, který obsahuje všechny data analýzy. Z tohoto důvodu se ukázala jako výborná volba právě možnost použití No-SQL databáze, která přijímá a podporuje ukladání tohoto typu dat bez nutnosti transformace tohoto objektu na jednotlivé tabulky relační databáze. 

\subsection{Požadavek 36}
Zde se jedná zejména o upřesnění, že požadavek tuto funkcionalitu podporovat nebude. Formuláře aplikace z pohledu textové tabulky i grafického náhledu není úplně možné zobrazit na mobilním zařízení v čitelné formě. Navíc to nedává ani smysl z pohledu samotné tvorby analýzy. 


\chapter{Rešerše existujích řešení}
\label{sec:nastroje}
Tato kapitola se bude zabývat popisem a srovnáním aktuálně dostupných nástrojů pro tvorbu FMEA. Nejdříve budou krátce představeni tři zástupci aplikací. Následně budou tito tři zástupci srovnání na základě zobecnění základních požadavků na nástroj, který by měl vzniknou v rámci této diplomvé práce. 

Možností pro vypracovná analýzy je více. Nejjednodušší variantou je použití předem definované šablony v libovolném tabulkovém editoru. Tento způsob nabízí velice základní funkce pro vypracování FMEA. Proto je lepší přistuoupit k nějakému profesionálnějšímu řešení, které výrazým způsob usnadňuje a zefektiňuje daný proces. Všechny uvedené nástroje byly vyzkoušeny na základě poskytované demo verze. Při záverečném srovnání bylo bráno v potaz, že některé funkce jsou dostupné až v placených verzích.

\section{Dostupné řešení}
\subsection{FMEA Studio(iQA system)}

\begin{figure}[h]
\centering
	\includegraphics[width=1.0\textwidth]{Figures/iqa.PNG}
	\caption{Nástroj FMEA Sudio }
	\label{fig:iqa}
\end{figure}

FMEA Studio slouží jako nadstavba výše zmíněného způsobu tvorby analýzy pomocí tabulkového editoru. Tento nástroj slouží jako rozšíření například do programu Microsfot Excel, kde pak nabízí například několik různých šablon zaměřených zejména na Návrh a Proces. Samozřejmostí je také automatický výpočet RPN a určení AP. Uživatel má také možnost nastavení vlastních měřítek pro hodnocení atributů Význam, Výskyt a Detekce. Dokonce je možné zobrazit relace mezi daty i pomocí stromové struktury. Další užitečnou funkcí je možnost nastavení priority jednotlivých hodnotících atributů pro jednodušší určení do jakého stavu identifikováné riziko přejde. Nástroj také nabízí základní zobrazení výstupu analýzy pomocí matice rizik a grafů. Formatování je umožněno zejména pomocí hostitelského programu. Ukázku tohoto rozšíření je možné vidět na obrázku \ref{fig:iqa}
\break
\break


\subsection{PQ-FMEA}
PQ-FMEA je čistě desktopová aplikace pro tvorbu FMEA. Tato aplikace umožňuje tvorbu dvou základních typů(DFEMA, PFMEA) a podporuje práci s typy RPFMEA, LFMEA, MFFMEA, SwFMEA, UFMEA. Aplikace nabízí docela široké možnosti, co se týká zobrazení dat. Tabulku lze zobrazit dle standardu AIAG nebo VDA nebo jejich společné variantě. Dále lze také zobrazit kontrolní plán, což je dokument, který se z pravidla vypracovává po ukončení práce na analýze zaměřené na proces. Tento nástroj také disponuje zajímavou možností tvorby analýzy v grafickém režimu. V prvé řadě lze přepínat mezi třemi módy, které určují krok analýzy. Po výběru módu lze tvořit relace mezi prvky stylem drag\&drop. U prvního módu reprezentující strukturální analýzu je tento způsob velice účinný. Nicméně u dalších dvou módu reprezentující funkční analýzu a analýzy selhání se zobrazují v rámci sloupců kombinace atributů z předchozích kroků a celý proces je tak velice nepřehledný. Také zobrazení pomocí tabulky není úplně nejpřívětivější. V rámci kroků analýzy nejsou nijak barevně rozlišeny prvky, které spolu souvisí a pro zobrazení relací chybí slučovaní buněk tabulky. Na druhou stranu tato aplikace nabízí velikou škálu vstupních a výstupních funkcí jako je:
\begin{itemize}
    \item Nastavení vlastních měřítek pro určení hodnotících atributů spolu s možností jejich importu a exportu
    \item Tvorba logů ze setkání a přidělování úkolu jednotlivým uživatelům
    \item Zobrazení výstupních grafů analýzy, matice rizik, rozložení rizik podle AP 
    \item Tisk jednotlivých částí analýzy
    \item Možnost vyhledávání v tabulce podle klíčového slova
    \item Podporu několika světových jazyků
\end{itemize}

Jedná se tak o docela robustní nástroj s širokou škálou funkcí nicméně podle názoru autora této práce s několika základními nedostaky. Ukázku této aplikace je možné vidět na obrázku \ref{fig:pq}

\begin{figure}[h]
\centering
	\includegraphics[width=1.0\textwidth]{Figures/pq.jpg}
	\caption{Nástroj PQ-FMEA }
	\label{fig:pq}
\end{figure}

\subsection{Relyence FMEA}
U předchozích dvou nástrojů se jednalo o desktopová řešení, kterým chyběla přímá podpora pro možnost sdílené práce na analýze více uživatelů. Tento problém by šel případně vyřešit za pomocí externích programů pro sdílení plochy jednoho z uživatelů a také samostatného sdílení souborů analýzy. Správným směrem se vydala společnost Relyence, která vytvořila aplikaci webovou. Tato apliakce nabízí kromě tvorby FMEA analýzy také několik dalších method z oblasti Risk Managementu jako je například Fault Tree, Reliabity Prediction apod. V rámci FMEA aplikace podporuje typy DFMEA, PFMEA, FMEA-MSR, FMECA. Tyto typy ovšem neodpovídají standardu AIAG/VGA pro automobilový průmysl. Aplikace sice nedisponuje grafickým zobrazením dat analýzy, ale nabízí opravdu komplexní možnosti práce s tabulkou. Také je možné grafické zobrazení a tvorba podpůrných souvisejích artefaktů jako je Boundary diagram nebo P-diagram. Samozřejmostí je také import a export dat analýzy do několika podporovaných formátů. Tento nástroj, narozdíl od dvou předchozích, určitým způsobem řeší správu uživatelů. Je zde možné si vytvořit uživatelský účet a ukládat si data analýzy. Zajímavostí je, že webová aplikace by měla být podle všeho také podporovat zobrazení na zařízení jako je tablet nebo dokonce mobilní zařízení. Ani u této apliakce ovšem nevypadá, že by existovali nějaké sofistikovanější možnosti společné práce na analýze. Ukázku této aplikace možné vidět na obrázku  \ref{fig:relyence}

\begin{figure}[h]
\centering
	\includegraphics[width=1.0\textwidth]{Figures/relyence.jpg}
	\caption{Nástroj Relyence FMEA }
	\label{fig:relyence}
\end{figure}

\section{Srovnání dostupných řešení}
V této části kapitoly budou srovnány výše uvedené nástroje pro tvorbu FMEA, které jsou dostupné na trhu. Součástí tohoto srovnání bude i nástroj, který byl vytvořen v rámci této diplomové práce. Důvodem je ukazát, jak byly jednotlivé požadavky na nástroj realizovány v porovnání s ostatními komerčními produkty. Dále následuje tabulka \ref{tab:compare}, kde lze vidět na základě jakých požadavků byly nástroje posouzeny a jak je jednotlivé řešení naplňují.
\break
\break
\break
\break
\break
\break

\begin{longtable}{|p{4cm} | p{12cm} |} 
        \caption{Srovnání dostupných řešení}
\label{tab:compare}
         \hline
& \textbf{1. Tvorba analýzy podporující typy DFMEA, PFMEA,...} \\ \hline
 FMEA Studio &	DFMEA,PFMEA, Software FMEA  \\ 
 PQ-FMEA &	 DFMEA,PFMEA  \\ 
 Relyence FMEA &	 DFMEA,PFMEA, FMECA, FMEA-MSR  \\ 
Vlastní řešení &	 DFMEA,PFMEA \\ \hline

& \textbf{2. Zobrazení dat analýzy pomocí textové tabulky.}  \\ \hline
 FMEA Studio &	Rozšíření tabulkového editoru + přidané některé vlastní funkce, správné slučování buněk, nedostatečné barevné rozlišení souvisejících atributů.  \\ 
 PQ-FMEA &	 Různé možností zobrazení, špatné slučování buněk, nedostatečné barevné rozlišení souvisejích atributů.  \\ 
 Relyence FMEA &	 Pokročilé funkce práce s tabulkou, správné slučování buněk na základně relací mezi atributy.  \\ 
Vlastní řešení &	 Zobrazení a následně i editace dat tabulky, zobrazení relací slučováním buněk, rozšířené barevné rozlišení souvosejích atributů s možností označení jednotlivých prvků ze strukturální analýzy. \\ \hline

& \textbf{3. Zobrazení dat analýzy v grafickém režimu.} \\ \hline
 FMEA Studio &	 Možnost zobrazit strukturu dat v jednotlivých krocích na samostatném panelu, přidávání i editace prvků.\\ 
 PQ-FMEA & Náhled i tvorba pomocí tří módu zobrazení stromovou strukturou, lehce nepřehledné v rámci některých módů, tvorba stylem drag\&drop a úpravá prvků.  \\ 
 Relyence FMEA &	 NE  \\ 
Vlastní řešení &	 Možnost tvorby stromové struktury, následně přidávání funkcí a selhání jednotlivým prvků ze struktury pomocí modálních oken, zoom, skrytí/zobrazení potomků. \\ \hline

& \textbf{4. Soubežná práce více uživatelů.} \\ \hline
 FMEA Studio &	Za pomocí externího software + sdílení souboru analýzy.  \\ 
 PQ-FMEA &	 Za pomocí externího software + sdílení souboru analýzy. \\ 
 Relyence FMEA &	  Základní práce s uživatelskými účty, jinak pomocí externího SW + sdílení souborů.  \\ 
Vlastní řešení &	 Možnost připojení několika uživatelů do skupin na základě stejného odkazu a sdílení prováděných změn v reálném čase. Registrace a přihlašování uživatelů. \\ \hline

& \textbf{5. Možnost export a importu dat analýzy. } \\ \hline
 FMEA Studio &	Pouze ukládání a načítaní souboru tabulkového editoru.  \\ 
 PQ-FMEA &	  Ukládání a načítání dat analýzy do vlastního formátu, import a export nastavení vlastních měřítek pro hodnocení.   \\ 
 Relyence FMEA &	 Ukládání a načítání dat analýzy do vlastního formátu.  \\ 
Vlastní řešení &	 Ukládání a načítaní dat analýzy do JSON formátu, export do formátu .xls, .png.\\ \hline

& \textbf{6. Další podpůrné funkce(nastavení vlastních měřítek hodnocení, grafy na výstupu)} \\ \hline
 FMEA Studio &	Nastavení vlastních měřítek pro hodcnocení, hranic pro hodnocení rizik, výstupní grafy a matice rizik, flow diagram,...  \\ 
 PQ-FMEA &	 Nastavení vlastních měřítek pro hodcnocení, výstupní grafy a matice rizik, tisk formulářů, logování a přidělování úkolů uživatelům.  \\ 
 Relyence FMEA &	  Možnost importu a tvorby několika souvisejících diagramů. \\ 
Vlastní řešení &	  Nastavení vlastních měřítek pro hodnocení, tvorba logů ze setkání, kontrola dokončení všech nalezených rizik. \\ \hline
\end{longtable}

Závěrem tohoto popisu lze říci, že ač nalezené nástroje obsahují několik dostupných typů FMEA, různých možností nastavení projektů, výstupních grafů a matic, tak postrádají jednu ze základních funkcí a to umožnit týmu provádějící analýzu souběžnou a efektivní práci na analýze přímo skrze používáný nástroj. To je nesporná výhoda nástroje, který byl vytvořen v rámci této práce. Další výhodou je, že vytvořený nástroj celkem jasně sdružuje atributy, které spolu souvisí, kdy u jiných nástrojů nejsou jednotlivé souvislosti hned na první pohled jasné. Následuje kapitola \ref{sec:navrh}, která již bude zaměřena vyvýjený nástroj, nejdříve jeho návrh a architekturu. 
 
%\input{Chapters/Introduction.tex}
%\input{Chapters/SampleChapter1.tex}
%\input{Chapters/SampleChapter2.tex}
%\input{Chapters/TechnicalDetails.tex}
%\input{Chapters/Conclusion.tex}

% Seznam literatury
\printbibliography[title={Liter`atura}, heading=bibintoc]

% Prilohy
\appendix
%\input{Chapters/Appendix1.tex}
%\input{Chapters/Appendix2.tex}

% Priloha vlozena primo do hlavniho LaTeX souboru. Ne vsechny prilohy je nutne mit ve zvlastnich souborech.
%\chapter{Dlouhý zdrojový kód}
%\lstinputlisting[label=src:CppExternal,caption={Dlouhý zdrojový kód v jazyce C++ načtený s externího souboru}]{SourceCodes/ArraySortingAlgorithms.cpp}

\end{document}
