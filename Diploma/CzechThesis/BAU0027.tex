% Nejprve uvedeme tridu dokumentu s volbami
\documentclass[czech,master]{diploma}
% Dalsi doplnujici baliky maker
\usepackage[autostyle=true,czech=quotes]{csquotes} % korektni sazba uvozovek, podpora pro balik biblatex
\usepackage[backend=biber, style=iso-numeric, alldates=iso]{biblatex} % bibliografie
\usepackage{dcolumn} % sloupce tabulky s ciselnymi hodnotami
\usepackage{subfig} % makra pro "podobrazky" a "podtabulky"
\usepackage[cpp]{diplomalst}

% Zadame pozadovane vstupy pro generovani titulnich stran.
\ThesisAuthor{Marek Bauer}

\ThesisSupervisor{Ing. Jan Kožusznik, Ph.D.}

\CzechThesisTitle{Nástroj pro tvorbu FMEA analýzy}

\EnglishThesisTitle{Tool for FMEA Analysis}

\SubmissionYear{2022}

\ThesisAssignmentFileName{ThesisSpecification_BAU0027.pdf}

% Pokud nechceme nikomu dekovat makro zapoznamkujeme.
\Acknowledgement{Rád bych na tomto místě poděkoval všem, kteří mi s prací pomohli, protože bez nich by tato práce nevznikla.}

\CzechAbstract{}

\CzechKeywords{}

\EnglishAbstract{}

\EnglishKeywords{}

\AddAcronym{DVD}{Digital Versatile Disc}
\AddAcronym{TNT}{Trinitrotoluen}
\AddAcronym{UML}{Unified Modeling Language}
\AddAcronym{HTML}{Hyper Text Markup Language}
\AddAcronym{TUG}{\TeX{} Users Group}

\addbibresource{biblatex-examples.bib}
\addbibresource{coffee.bib}

% Novy druh tabulkoveho sloupce, ve kterem jsou cisla zarovnana podle desetinne carky
\newcolumntype{d}[1]{D{,}{,}{#1}}


% Zacatek dokumentu
\begin{document}

% Nechame vysazet titulni strany.
\MakeTitlePages

% Jsou v praci obrazky? Pokud ano vysazime jejich seznam a odstrankujeme.
% Pokud ne smazeme nasledujici dve makra.
\listoffigures
\clearpage

% Jsou v praci tabulky? Pokud ano vysazime jejich seznam a odstrankujeme.
% Pokud ne smazeme nasledujici dve makra.
\listoftables
\clearpage

% A nasleduje text zaverecne prace.
\chapter{Úvod}
\label{sec:Uvod}
V mnoha odvětvích lidské činnosti je možné přijít do styku s nedokonalostmi v návrzích produktů nebo výrobních procesech, které mohou vést k nezamýšlenému chování a více či méně važným následkům. Následky mohou být triviální, které výrazným způsobem neomezují funkci produktu nebo naopak vážné, které mohou být životu nebezpečné. Dalším důsledkem špatného technologického procesu nebo návrhu je samozřejmě také finanční stránka, kdy nadměrná produkce disfunkčních dílů stojí výrobce peníze navíc za materiál nebo při objevení závady až u koncové zákazníka také výdaje s vyřizováním reklamací, opravami a podobně. Tyto důvody vedly k vytvoření procesů, metod a norem, které mají za cíl odhalení, odstranění nebo alespoň zmírnění možných závad a rizik ještě před začátkem realizace dané činnosti. Souhrně lze tyto snahy označit Risk Management, tedy disciplínu zabývající se správou a řízením rizik. 

Jednou z těchto metod je analytická metoda FMEA(Failure Mode and Effects Analysis), kterou se zabývá tato diplomová práce. Následující kapitola  \ref{sec:FMEA} bude obsahovat základní popis této metody spolu s její historií, dále popis v jakých oborech se metoda aktuálně nejvíce uplatňuje a také rozdělení na základní typy podle toho v jaké fázi vývoje produktu se analýza provádí.

Kapitola \ref{sec:FMEA_postup} se budou zabývat tím, jaký je postup při vypracovávání této analýzy. Analýza se skládá celkem ze sedmi kroků, které budou podrobně vysvětleny. Budou zde také uvedeny výňatky tabulky z formulářů, ve kterých se analýza provádí. Na jednotlivých krocích analýzy budou zobrazeny odlišnosti dvou typů analýz, na které je práce zaměřena.

Následovat bude kapitola \ref{sec:pozadavky} sloužící jako popis požadavků na vytvářený nástroj. Specifikace požadavků bude inspirována disciplínou inženýrství požadavků. Požadavky budou členěny dle typů a kapitol, budou obsahovat popis a také jejich konečný stav, podle toho jak se vyvýjel v rámci průběhu návrhu a implementace. Některé zajímavé požadavky budou popsány detailněji z důvodu vysvětlení důvodu jejich zařazení nebo naopak nezařazení do celkové sady požadavků na vyvýjený nástroj. Tento seznam požadavků vychází jak ze samotného zadaní diplomové práce, tak z konzultací s vedoucím práce a také rešerše existujících řešení, kterou se bude zabývat kapitola \ref{sec:nastroje}. 

Rešerše existujících řešení je cíleně umístěna až za kapitolou zabývající se analýzou a specifikací požadavků z toho důvodu, že budou v této kapitole srovnány jednotlivé nástroje na základě zjednodušené sady požadavků vzniklé v předchozí kapitole. Součástí srovnání bude i nástroj vytvořený autorem této práce, kde bude vidět srovnání, jak byly jednotlivé požadavky naplněny oproti profesionálním řešením. 

Další kapitola číslo \ref{sec:navrh} se již bude zabývat vlastním nástrojem, který byl vytvořen v rámci této diplomové práce. Bude obsahovat popis architektury, použitých technologií, návrh databáze a uživatelského rozhraní. Budou zde uvedeny i důvody proč byly jednotlivé návrhové rozhodnutí učiněny. 

Popisem implementace vytvořeného nástroje se bude zabývat kapitola \ref{sec:implementace}. Implementace bude rozdělena na klientskou a serverovou část. Popis bude brán jak z uživatelského, tak implementačního hlediska. Budou zde podrobněji ukázány jednotlivé vlastnosti a funkce nástroje. Konec kapitoly se bude zabývat testováním vytvořeného nástroje ve formě integračních testů. 

Závěrem bude hodnocení dosažených výsledků a vytvořeného nástroje. Budou zde také uvedeny různé možnosti rozšíření nástroje, které z různých důvodu nebyly implementovány. 
\endinput
\chapter{FMEA}
\label{sec:FMEA}
FMEA je zkratka z anglického výrazu Failure Mode and Effect Analysis, česky lze toto označení přeložit jako Analýza výskytu vad a jejich dopadů nebo Analýza příčin a důsledků. Jedná se tedy o analytickou metodu, která ma za cíl odhalení možných vad ve výrobním procesu nebo návrhu produktu, nalezení příčin výskytu těchto vad, ohodnocení závažnosti daného rizika a snaha o jeho zmírnění nebo odstranění. Na vypracování této analýzy se většinou podílí tým odborníků z různých oblastí daného odvětví, kteří využají svých znalostí a zkušeností ze svých profesí k odbornému posouzení problémů v různých fázích analýzy. Obvykle tento tým tvoří vedoucí výroby, zástupce z oddělení pro kontrolu kvality, konstruktér, technolog popřípadě další odborníci. Mezi relevantní účastníky také může patřit zákazník, kterému mohou být poskytovány výstupy z jednotlivých verzí analýzy. V tomto případě může být FMEA součástí smluvní dohody mezi výrobcem a zákazníkem jako záruka kvality například v rámci PPAP (Production Part Approval Process). Přesto, že je FMEA prováděna před začátkem realizace dané fáze vývoje, je dobré přistupovat k dokumentům obsahujícím analýzu jako k dynamickým a v čase zlepšovat jejich kvalitu a přesnost.  

\section{Historie FMEA}
\label{sec:historie}
Počátky FMEA sahají do 40. let minulého století, kdy byla FMEA poprvé použita americkou armádou pro redukci potenciálních závad při výrobě munice. Metoda se ukázala jako vysoce efektivní a kolem roku 1960 ji začlenila do svých přípravných technik i společnost NASA. FMEA se ukázala jako podstatná součást mise Apollo. Od 70.let 20. století již následoval automobilový průmysl, který tvoří jedno z hlavních odvětví, kde je tato metoda využívána. Prvotním uživatelem byla automobilka Ford, která přijala FMEA jako reakci na špatný bezpečnostní stav jejich modelu Ford Pinto. Jejich příkladu pak následovali další američtí i evropští výrobci. Tyto události vedly ke vzniku asociací AIAG a VDA, které definují standardy pro zvýšení kvality pomocí nástrojů jako je FMEA, SPC(statistical process controll) nebo MSA(Measurement system analysis). Z důvodu rozšířenosti použití FMEA analýzy v tomto průmyslu bude i v následujících popisech a příkladech brán pohled na tvorbu FMEA z toho odvětví.  

\section{Aplikace FMEA}

FMEA se člení zejména na tři typy, které se odvíjejí podle fáze vývoje. Jedná se o tyto tři druhy: 

\begin{itemize}
	\item  DFMEA (Design)
	\item  PFMEA (Process)
	\item  SFMEA (System)
\end{itemize}
V této diplomové práci bude věnována pozornost hlavně analýze zaměřené na návrh a proces. Tyto dvě zaměření jsou v praxi využívány nejvíce pravděpodobně díky tomu, že se zaměřují na specifičtější části vývoje produktu. Součástí FMEA analýzy je dekompozice subjektu ve fázi strukturální analýzy na nízkoúrovňové části, pro které se lépe hledájí možné způsoby selhání. Použití metody na celý systém se tak může zdát jako zbytečně komplexní řešení a v některých případech tato možnost ani nepřipadá v úvahu. Konkrétně je dobré uvést příklad z rozsáhlého autombilového průmyslu, kde se výroba vozu skládá ze spolupráce několika různých výrobců komponent, kteří často nemusí mít přehled o práci ostatních dodavatelů. V některých případech jsou výrobky součástí většího celku a těžko se při použití metody DFMEA nalézají všechny možné rizikové scénáře. V těchto případech se například provádí pouze analýza zaměřená na konkrétní výrobní proces(PFMEA), u kterého lze určit možná rizika. 

\subsection{DFMEA}
\label{subsec:DFMEA}
DFMEA se používá pro analýzu nových návrhu produktů. Měla by tedy navazovat na ukončení fáze návrhu a vycházet tak ze softwarových artefaktů z toho vyplývajících. Jedním z těchto artefaktů může být například blokový diagram(Boundary diagram), který pokrývá určitou část systému. Pomocí blokového diagramu lze zobrazit rozhraní a vztahy mezi jednolivými komponentami. Mezi další podklady pro tvorbu DFMEA může patřit: 
\begin{itemize}
	\item  Normy a předpisy
	\item  Fyzikální specifikace materiálů
	\item  Návrhy obdobných produktů, popř. DFMEA pokud byla provedena
	\item  Požadavky zákazníka
	\item  Diagram parametrů (P-diagram)
	\item  Shrnutí všech požadavků na návrh
	
\end{itemize}

Stejně jako u ostatních typů FMEA je i zde potřeba průběžně aktualizovat softwarové artefakty i při přechodu do dalších fází vývoje produktu. Postup pro vypracovaní DFMEA by se dal rámcově shrnout do těchto bodů: 
\begin{enumerate}
	\item Identifikace rizik
	\item Analýza rizik
	\item Hodnocení a řízení rizik 
	\item Návrh optimalizačních opatření
	\item Přehodnocení rizik
\end{enumerate}
Podprobnějším popisem struktury a kroků, které se provádí při vypracování DFMEA se bude zabývat kapitola \ref{sec:FMEA_postup}

\subsection{PFMEA}
\label{subsec:PFMEA}
PFMEA se používá pro analýzu nových procesů zejména z pohledu výroby. Dá se říci, že by měla navazovat ne předchozí analýzu návrhu produktu, kdy výrobní proces je dalším logickým krokem při vývoji. Mezi další vstupní předpoklady patří například flow diagram, který dekomponuje proces do série nazujících kroků. V praxi je možné setkat se i s případem, kdy se PFMEA provádí až jako snaha o zlepšení stávajícího procesu. V tomto případě lze jako podklad pro vypracování také vzít v úvahu fyzickou prohlídku výrobní haly. Tuto analýzu často vykonává stejný tým jako v případě DFMEA.  



\endinput
\chapter{Vypracování FMEA }
\label{sec:FMEA_postup}
V této kapitole bude podrobněji popsán postup při použití metody FMEA pro návrh a proces. Obě tyto varianty mají totožnou strukturu a tak budou její jednotlivé části popsány zároveň pro vytvoření obrazu v jakých atributech se konkrétní části obou typů analýz liší. 

Výňatky z formulářů odpovídají standardu příručky FMEA pro automobilový průmysl\cite{fmeaHandbook} Jednotlivé reprezentace FMEA se mohou lišit, zejména při použití v odlišných odvětvích. Jak již bylo naznačeno v kapitole \ref{sec:historie}, tak lze analýzu aplikovat v odvětvích jako je armádní průmysl, zdravotnictví, kosmický průmysl a mnoho dalších. Nicméně nejrozšířenějším odvětvím, který využívá metody FMEA, je automobilový průmysl a i proto budou uvedené příklady z tohoto odvětví. Také je možné se setkat s některými volitelnými atributy, které nemají veliký význam a nebude na ně brán v následujícím popisu zřetel.  

\section{Plánování a příprava(1. krok)}
V kapitolách \ref{subsec:DFMEA} a \ref{subsec:PFMEA}  byly zmíněny předpoklady a vstupní podmínky před začátkem samotné analýzy. V prvník kroku je také sestaven tým odborníků, kteří budou za pomocí společných setkání provádět samotnou analýzu. V týmu je potřeba před začátkem práce definovat základní pravidla a pojmy. Je potřeba definovat škálu pro atributy, pomocí kterých bude probíhat hodnocení nalezených rizik. Od této škály se například bude odvíjet, od jaké hodnoty bude akceptována určitá míra rizika. Tyto atributy jsou: 

\begin{enumerate}
	\item Severity (Význam)
	\item Occurance (Výskyt)
	\item Detection (Odhalitelnost)
\end{enumerate}

Pro každou z těchto tří atributů je potřeba stanovit škálu, která nabývá hodnot od jedné do desíti a každé hodnocení obsahuje slovní popis této hodnoty pro jednodušší určení hodnoty. 

Po stanovení těchto atributů je možné přistoupit k vyplnění úvodních informací tvořící hlavičku analýzy. 

\begin{center}
\begin{table}[h]
	\centering
	\caption{Hlavička analýzy}
 \label{tab:header_FMEA}
	\label{tab:Head1}
        \begin{tabular}{|c | r | l |} 
         \hline
 \multicolumn{3}{|c|}{Planning and preparation (Step 1)} \\

         \hline
         1 & Company Name & ABC  \\ [0.5ex] 
         \hline
         2 & Location & Bratislava  \\ [0.5ex] 
         \hline
         3 & Customer Name & VW Group  \\ [0.5ex] 
         \hline
         4 & Model Year / Program(s) & 2006/J77 \\ [0.5ex] 
         \hline
         5 & Subject & J77 Instrument Cluster \\ [0.5ex] 
         \hline
         6 & Responsibility & John Doe (Product Eng.) \\ [0.5ex] 
         \hline
         7 & D(P)FMEA Start Date & 1.1.2023  \\ [0.5ex] 
         \hline
         8 & D(P)FMEA Revision Date & 27.2.2023 \\ [0.5ex] 
         \hline
         9 & D(P)FMEA ID Number & 77 \\ [0.5ex] 
         \hline
         10 & Confidentiality Level & Proprietary \\ [0.5ex] 
         \hline
         11 & Cross-Functional team & John Doe, Henry Smith \\ [0.5ex] 
         \hline
        \end{tabular}
    \end{table}
\end{center}

Tabulka \ref{tab:header_FMEA} obsahuje seznam atributů, které se vyplňují v rámci úvodní fáze analýzy. Zde je popis jednotlivých atributů:

\begin{enumerate}
	\item \textbf{Název společnosti} - jedná se o společnosti v rámci, které je analýza prováděna
	\item \textbf{Lokace} - zde může lokace odpovídat buď lokací inženýrského týmu zodpovědného za návrh produktu(DFMEA) nebo umístěním výrobního závodu(PFMEA) 
	\item \textbf{Jméno zákazníka} - Podle zaměření analýzy, může být zákazníkem jak koncový uživatel, tak i interní oddělení, které navazuje ve výrobním cyklu
 \item \textbf{Modelový rok, Program} - specifikace modelu produktu, program(DFMEA) nebo platforma(PFMEA)
 \item \textbf{Předmět} - vysokoúroňový popis zkoumaného prvku
 \item \textbf{Zodpovědnost} - vlastník dokumentu analýzy, často v roli manažera setkání týmu
 \item \textbf{Začátek analýzy} - počáteční datum
 \item \textbf{Datum revize} - datum kontroly provedených změn 
 \item \textbf{Identifikační číslo analýzy} - jedná se o interní hodnotu, má smysl v rámci kontextu firmy, kdy může sloužit jako odkaz na dokument s analýzou
 \item \textbf{Stupeň důvěrnosti} - může nabývat hodnot (obchodní, proprietární, důvěrný)
 \item \textbf{Multioborový tým} - seznam lidi podílejících se na tvorbě analýzy
\end{enumerate}

\section{Analýza struktury(2. krok)}
\label{sec:FMEA_postup_2}
Cílem strukturální analýzy je dekompozice zkoumaného produktu nebo procesu na menší části, které pak slouží jako vstupní parametry v dalších krocích. Dekompozice slouží pro jednodušší pochopení zkoumaného produktu. Výsledkem je nejčastěji stromová struktura o výšce stromu tři, kdy kořenem stromu je přímo zkoumaný prvek.





\begin{center}
\begin{table}[h]
	\centering
	\caption{Formulář pro analýzu struktury(DFMEA) }
	\label{tab:structure_DFMEA}
\begin{tabular}{ |p{4cm}|p{3cm}|p{3cm}|  }
 \hline
 \multicolumn{3}{|c|}{Structure Analysis (Step 2)} \\
 \hline
 1. Next Higher Level& 2.Focus Element
No. And Name of Focus Element &3. Next Lower Level of Characteristic Type\\
 \hline
 Instrument Cluster B90   & 6. Speed clock    &Stepper motor\\


 \hline
\end{tabular}\  
\end{table}
\end{center}

\begin{center}
\begin{table}[h]
	\centering
	\caption{Formulář pro analýzu struktury(PFMEA) }
	\label{tab:structure_PFMEA}
\begin{tabular}{ |p{4cm}|p{3cm}|p{3cm}|  }
 \hline
 \multicolumn{3}{|c|}{Structure Analysis (Step 2)} \\
 \hline
1. Process Item
System, Subsystem, Part Element or Name of Process
& 2. Process Step
No. And Name of Focus Element
&3. Process Work Element

4M Type\\
 \hline
X98 sun roll assembly line   & 020 Assembly body-mechanism   &Operator\\


 \hline
\end{tabular}\  
\end{table}
\end{center}

V tomto kroku analýzy se nejvíce projevují rozdíly mezi jejími dvěma zmíněnými typy. Rozdílnost atributů, které jsou v této fázi vyplňovány je možné porovnat v tabulkách \ref{tab:structure_DFMEA} a  \ref{tab:structure_PFMEA} 
Dále následuje popis jednotlivých atributů, které se vyplňují v rámci strukturální analýzy. 

\begin{enumerate}
	\item \textbf{Vyšší úroveň / Položka procesu systému, subsystému, dílu nebo název procesu} - Prvek na nejvýšší úrovni v rámci předmětu analýzy, pro tento prvek se v se v rámci analýzy selhání definují důsledky selhání
	\item \textbf{Vybraný prvek, číslo a označení/Krok procesu, číslo a označení} - Prvek na druhé úrovni, pro tento prvek se v rámci analýzy selhání určuje jeden ze základních pílířů analýzy, a to způsob selhání
	\item \textbf{Nižší úroveň nebo druh charakteristiky/Prvek provádění činnosti, 4M} - Prvek na nejnižší úrovni, pro který se určují příčiny selhání. V analýze zaměřené na proces je zde uveden návodně typ 4M, který vyjadřuje(Machine, Man, Material, Milieu), tedy určité kategorie, do kterých by mohl hodnotící tým zařadit prvek na této úrovni. Další kategorie mohou také být Method, Measurement.  
\end{enumerate}

\section{Analýza funkcí(3. krok)}
\label{sec:FMEA_postup_3}
V rámci analýzy struktury byly vytvořeny objekty, pro které ve fázi analýza funkcí bude tým definovat jejich funkce. Nalezené funkce budou dále sloužit jako vstupní parametry pro analýzu selhání. 

Stejně jako v předchozím kroku, kdy výsledná analýza tvořila stromovou strukturu, neboli jednotlivé objekty měli mezi sebou definované relační vazby, tak je tomu obdobně i v této části analýzy. Zde je kladen důraz hlavně na funkcionalitu prvků na druhé úrovni. Pro tyto funkce se pak hledají funkce ze třetí úrovně, které zajišťují jejich funkcionalitu. Stejně je potřeba nalézt funkce na první úrovni, které jsou chápany jako výsledky funkce ze druhé úrovně. 

Analýza funkcí je dalším příkladem, kdy se jednotlivé typy analýzy zaměřené na návrh a proces liší. Tyto odlišnosti znázorňují tabulky \ref{tab:function_DFMEA} a \ref{tab:function_PFMEA} 

\begin{center}
\begin{table}[h]
	\centering
	\caption{Formulář pro analýzu funkcí(DFMEA) }
	\label{tab:function_DFMEA}
\begin{tabular}{ |p{5cm}|p{4cm}|p{4cm}|  }
 \hline
 \multicolumn{3}{|c|}{Function Analysis (Step 3)} \\
 \hline
 1. Next Higher Level Function and Requirement &
2. Focus Element
Function and Requirement &
3. Next Lower Level Function and Requirement or Characteristic\\
 \hline
 Display vehicle status: spped, torq, gas level, engine temperature..   & Display vehicle speed    & Convert a train of input pulses into a precisely defined increment in the shaft position\\


 \hline
\end{tabular}\  
\end{table}
\end{center}

\begin{center}
\begin{table}[h]
	\centering
	\caption{Formulář pro analýzu funkcí(PFMEA) }
	\label{tab:function_PFMEA}
\begin{tabular}{ |p{5cm}|p{4cm}|p{4cm}|  }
 \hline
 \multicolumn{3}{|c|}{Function Analysis (Step 3)} \\
 \hline
1. Function of the Process Item
Function of System, Subsystem, Part Element or Name of Process
& 2. Function of the Process Step and Product Characteristics
(Quantitative value is optional)
& 3. Function of the Process Work Element and Process Characteristics
\\
 \hline
Assembly body with mechanism   & Screw body with mechanism with 2 screws   &Take the first screw and insert into driver then press tool start button\\


 \hline
\end{tabular}\  
\end{table}
\end{center}

Popis atributů v rámci analýzy funkcí:
\begin{enumerate}
	\item \textbf{Funkce a požadavek vyšší úrovně / Funkce položky procesu Funkce systému, subsystému, dílu/komponentu nebo procesu} - Funkce odpovídající prvku nebo procesu na nejvyšší úrovni. Tomuto prvku může být přiřazeno více funkcí, ne pouze jedna jako v případě analýzy struktury.
	\item \textbf{Funkce a požadavek vybraného prvku/Funkce procesního kroku a charakteristika produktu(kvantitativní hodnota je volitelná)} - Funkce odpovídající prvku na druhé úrovni(subsystém, komponent) nebo kroku procesu(čeho musí daná stanice dosáhnout)
	\item \textbf{Funkce a požadavek nebo charakteristika nižší úrovně/Funkce prvku provádějící činnost a charakteristika procesu} - Funkce odpovídající prvku na nejnižší úrovni(komponent, díl) nebo prvku provádějící nějakou činnost v procesu
\end{enumerate}



\section{Analýza selhání(4. krok)}
Analýza selhání je klíčovým krokem v rámci celé anýlýzy. Z předchozí činnosti, kdy byly objektům přiřazeny funkce, je cílem tohoto kroku zjistit, jak by objekt o dané funkci mohl selhat. Pro nalezené selhání se dále hledají důsledky a příčiny, které jsou přiřazeny funkcím na první a druhé úrovni v rámci celkové struktury. 

Stejně jako u předchozích dvou kroků popsaných v kapitolách \ref{sec:FMEA_postup_2} a \ref{sec:FMEA_postup_3} existují mezi atributy relační vazby. Zde je celkem logicky každému selhání přiřazeno několik možných důsledků a příčin. Důsledky selhání jsou definovány jako dopad na celkový systém, a proto náleží kořenovému objektu a jedné jeho funkci. Příčiny selhání jsou definovány jako selhání prvků na nejnižší úrovni dekompozice a slouží spolu s ostatními atributy jako vstupní parametry pro analýzu rizik v následujícím kroku.

Analýza selhání je krok, při kterém se již dva zmiňované typy od sebe tolik neliší. Význam atributů je prakticky stejný, akorát se odkazují na různé objekty z analýzy struktury. Porovnání obou typů zobrazují tabulky \ref{tab:failure_DFMEA} a \ref{tab:failure_PFMEA}


\begin{center}
\begin{table}[htp]
	\centering
	\caption{Formulář pro analýzu selhání(DFMEA) }
	\label{tab:failure_DFMEA}
\begin{tabular}{ |p{4cm}|p{0.5cm}|p{3cm}|p{4cm}|  }
 \hline
 \multicolumn{4}{|c|}{Failure Analysis (Step 4)} \\
 \hline
  1. Failure Effects (FE) to the next higher level and/or End User&
  \begin{turn}{-90}Severity (S) of FE\end{turn} &
2. Failure Mode (FM) of the Focus Element &
3. Failure Cause (FC) of the Next Lower Level Element or Characteristic
\\
 \hline
 Speed not display.
  & 5
  & Speedometer needle lock to 0 position. & Lost connection between stepper motor and board due bad soldering.
\\
 \hline
\end{tabular}\  
\\
\hfill \break
\hfill \break
\centering
	\caption{Formulář pro analýzu selhání(PFMEA) }
	\label{tab:failure_PFMEA}
\begin{tabular}{ |p{4cm}|p{0.5cm}|p{3cm}|p{4cm}|  }
 \hline
 \multicolumn{4}{|c|}{Failure Analysis (Step 4)} \\
 \hline
  1. Failure Effects (FE)
&
  \begin{turn}{-90}Severity (S) of FE\end{turn} &
2. Failure Mode (FM) of the Process Step
 &
3. Failure Cause (FC) of the Work Element

\\
 \hline
\textbf{Plant:}
Gap between housing and mechanism more than tolerance - rework necessary (less than 100% of batch) 

\textbf{Ship to Plant:}
Additional insertion force to assembly it into door panel

\textbf{End user:}
Possibility in time to hear noises (disturbed)

  & 5
  & Missing one screw
 & Operator skip

\\
 \hline
\end{tabular}\  
\end{table}
\end{center}



\clearpage
Popis atributů v rámci analýzy selhání:
\begin{enumerate}
	\item \textbf{Následek (FE) na prvku vyšší úrovně/koncového uživatele } - Důsledek selhání odpovídající prvku na nejvyšší úrovni. Jak je vidět na uvedeném příkladu pro PFMEA, tak je možné držet se určitých mantinelů pro definování dopadů selhání. Konkrétně je možné určit dopady na vlastní továrnu, továrnu zákazníka nebo koncového uživatele. 
 
 Ve formulářích pro analýzu selhání je uveden i atribut Závažnost, který se ale hodnotí až v rámci analýzy rizik. Proto mu bude věnováno více pozornosti až v kapitole \ref{sec:FMEA_postup_5}
	\item \textbf{Vada(FM) vybraného prvku/ Vada(FM) kroku procesu} - Selhání(vada) prvku nebo procesního kroku na druhé úrovni. Jedná se o hlavní atribut v rámci celé analýzy. Tým stanovující  selhání může vycházet například z různých předpokladů nebo obdobných systému a jejich FMEA(pokud byla provedena). Stanovení hodnocení tohoto atributu je důležité i z hlediska názvosloví a je potřeba používat odbornou a ucelenou terminologii.  
	\item \textbf{Přičina(FC) na prvku nižší úrovně nebo charakteristiky/Přičina(FC) v prvku provádějící činnost} - Přičina selhání odpovídající prvku na nejnižší úrovni. V případě procesní analýzy odpovídá prvku z kategorie 4M. 
\end{enumerate}


\section{Analýza rizik(5. krok)}
\label{sec:FMEA_postup_5}
Analýza rizik vychází z předchozího kroku, kdy hodnotící tým stanovil možné způsoby selhání,jejich příčiny a důsledky. V rámci analýzy tohoto kroku bude tým odborníků hodnotit toto riziko na základě aktuálně stanovených opatření. Výstupem hodnocení pro každé riziko musí být jedna z následujích variant:
\begin{itemize}
	\item Aktuální opatření jsou dostačující/hodnocení rizika není na tolik závažné, aby bylo nutné přistupovat k nějakým změnám. 
    \item Riziko může mít negativní dopady do té míry, že je potřeba závést nové opatření nebo upravit ty stávající tak, aby došlo ke zmírnění hodnocení některých atributů.
\end{itemize}
Analýza jako celek nemůže být uzavřena, dokud nejsou všechny rizika v jednom z uvedených stavů.

Tato fáze analýzy je pro oba typy FMEA prakticky totožná a význam atributů je stejný, proto bude uveden pouze jeden společný příklad.

\begin{center}
\begin{table}[h]
	\centering
	\caption{Formulář pro analýzu rizik }
	\label{tab:risk_FMEA}
\begin{tabular}{|p{4cm}|p{0.5cm}|p{4cm}|p{0.5cm}|p{0.5cm}|  }
 \hline
 \multicolumn{5}{|c|}{Risk Analysis (Step 5)} \\
 \hline
1. Current preventive control (PC) for FC
&
  \begin{turn}{-90}2. Occurence (O) of FC\end{turn} &
3. Current detection control (DC) for FC or FM
 &
  \begin{turn}{-90}4. Detection (D) of FC or FM\end{turn}
 &
  \begin{turn}{-90}5. D(P)FMEA AP\end{turn}

\\
 \hline
No prevention.
& 10
& Testing method to be developed.
& 10
& H


\\
 \hline
\end{tabular}\  
\end{table}
\end{center}


Popis atributů v rámci analýzy rizik:
\begin{enumerate}
	\item \textbf{Stávající preventivní opatření k příčině} - Příklad opatření, které firma aktuálně využívá a slouží jako prevence selhání. 
	\item \textbf{Výskyt příčiny} - Jeden ze tří hodnotících atributů. Nabývá hodnot nejčastěji ze stupnice 1-10. Číselnému hodnocení odpovídá také hodnocení slovní, které usnadňuje stanovení číselné hodnoty. Příklad tabulky, která se používá pro hodnocení výskytu selhání je součástí přílohy... Pro oba typy FMEA se používá odlišná tabulka. Jeden z aspektů pro hodnocení je také předchozí atribut. Dá se říci, že čím kvalitnější jsou stávající preventivní opatření, tím bude i hodnocení tohoto atributu menší.     
	\item \textbf{Stávající opatření pro detekci příčiny nebo selhání} - Metoda, kterou tým v rámci tohoto atributu definuje již není preventivní, ale slouží k odhalení už vzniklého selhání. 
	\item \textbf{Detekce příčiny} - Poslední hodnotící atribut. Stejně jako u atributu výskytu, tak dosahuje hodnot 1-10, které odpovídá i slovní popis. Taktéž navazuje na předchozí atribut, který také hraje roli v tom jak moc vysoká bude číselná hodnota detekce. 
	\item \textbf{Priorita (Action Priority) v rámci DFMEA nebo PFMEA} - 
 Ještě před popisem posledního atributu je potřeba zmínit atribut Význam, který byl součástí předchozí ukázky formuláře v tabulce \ref{tab:failure_DFMEA} nebo \ref{tab:failure_PFMEA}. Význam je první z hodnotících atributů, který má obvykle v rámci celkového hodnocení rizika největší váhu. Význam se váže na množinu dopadů odpovídající nějakému selhání tzn. že pro více hodnot Výskyt a Detekce může být použita společná hodnota atributu Význam. Hodnocení tohoto atributu probíhá stejným způsobem jako u Výskytu a Detekce, nicméně tento atribut vyjadřuje i určité dopady selhání nebo příčinny na koncového zákazníka. Tyto dopady mohou být triviální, kdy dochází například pouze k ovlivnění vzhledu, zvuku nebo vibracím, ale také velice závažným, kdy může být uživatel ohrožen na životě. Příkladem takové situace je, když je analyzovaným produktem nějaký důležitý komponent automobilu. Tabulka sloužící pro stanovení hodnoty je taktéž součástí příloh....
 
 Poslední atribut analýzy rizik slouží jako součást rozhodnutí, jak s odhaleným rizikem naložit. Tento atribut vychází ještě z jedné hodnoty, která se nicméně již v nových verzích analýzy neuvádí. Touto hodnotou je RPN(Risk Priority Number) a vyjadřuje jednoduše součin číselného hodnocení atributů Význam, Výskyt a Detekce. Logicky se bude tato hodnota pohybovat v rozmezí 1-1000. O výpočet této hodnoty se z pravidla stará použitý software. Na základě vypočtené hodnty RPN je stanovena i hodnota atributu AP, která může nabývat hodnot(High, Medium, Low) .

Jak již bylo řečeno AP je pouze součástí pro rozhodnutí do jakého stavu dané riziko přejde. Tím, že AP vychází ze součinu, jehož tři činitelé mají stejnou váhu, tak nemusí hodnota plně reflektovat stav daného rizika. Pro ucelené a konečné hodnocení je potřeba v rámci týmu vzít v úvahu o jakou konkrétni situaci se jedná a jak moc mají stanovené hodnoty v tomto případě váhu. 
 
\end{enumerate} 
\section{Optimalizace(6. krok)}
\label{sec:FMEA_postup_6}
Ná základě předchozího kroku tým vykonávající analýzu vyhodnotil pro jaké rizika je potřeba zavést dodatečné opatření pro zmírnění některého z atributů a snížení celkového hodnocení v rámci AP. Těchto rizik se bude týkat předposlední krok a to je Optimalizace. Zjednodušeně je cílem tohoto kroku stanovit nové opatření pro prevenci a detekci selhání, přiřadit provedení změn konkrétní osobě, určit datum revize a po uplnutí tohoto data provést opětovné hodnocení atributů SOD s předpokladem, že bude výsledná hodnota rizika zmírněna.

Tento krok je posledním v rámci formuláře, který má skupina hodnotitelů k dispozici. Stejně jako tomu bylo u několika předchozích kroků, tak je význam atributů v tomto kroku společný pro oba typy analýzy. Dále následuje tabulka \ref{tab:optimization_FMEA}, která obsahuje seznam všech atributů, které se v rámci tohoto kroku vyplňují.

\begin{landscape}
\begin{table}[h]
	\centering
	\caption{Formulář pro Optimalizaci }
	\label{tab:optimization_FMEA}
\begin{tabular}{|p{2.5cm}|p{2.5cm}|p{2.5cm}|p{2.5cm}|p{1.5cm}|p{2.5cm}|p{2.5cm}|p{0.5cm}|p{0.5cm}|p{0.5cm}|p{0.5cm}| }
 \hline
 \multicolumn{11}{|c|}{Optimization (Step 6)} \\
 \hline
1. D(P)FMEA Prevention Action &
2. D(P)FMEA Detection Action &
3. Responsible Persons Name &
4. Target Completion Date &
5. Status &
6. Action taken with Pointer to Evidence &
7. Completion Date &

\begin{turn}{-90}8. Severity(S)\end{turn} &
\begin{turn}{-90}9. Occurence (O)\end{turn} &
\begin{turn}{-90}10. Detection (D)\end{turn} &
\begin{turn}{-90}11. D(P)FMEA AP\end{turn}

\\
 \hline
Re-use same design as for X44 project - proved solution. &
Update test method - test to failure method.& 
John Doe & 
15.05.2020 &
C & 
Design Review 244243 May 27 2020 & 
15.05.2020 & 
5 & 
3 & 
3 & 
L 

\\
 \hline
\end{tabular}\  
\end{table}
\end{landscape}

Popis atributů v rámci optimalizace:
\begin{enumerate}
	\item \textbf{Preventivní opatření} - Zavedení nového preventivního opatření nebo úprava stávajícího
	\item \textbf{Opatření k odhalení} - Zavedení nového opatření pro detekci chyby nebo úprava stávajícího
	\item \textbf{Odpovědná osoba} - Přiřazení realizace změn v rámci opatření konkrétní osobě. Tato osoba by měla být součástí týmu a také uvedena ve skupině účastníků v hlavičce dokumentu.
	\item \textbf{Plánované datum dokončení}
	\item \textbf{Status} - Tento atribut udává v jakém stavu je realizace změn v opatřeních. Může nabývat například hodnot (O - Open, DP - Decision Pending, IP - Implementation Pending, C - Completed, NP - Not Implemented)
	\item \textbf{Přijatá opatření s odkazem na důkaz} - Dokumentace provedených změn. Měl by obsahovat i odkaz na další dokument popisující provedené opatření. 
	\item \textbf{Datum dokončení} 
	\item \textbf{Význam} - Opětovné hodnocení významu selhání
	\item \textbf{Výskyt} - Opětovné hodnocení výskytu selhání
	\item \textbf{Detekce} - Opětovné hodnocení detekce selhání
	\item \textbf{Priorita} - Po stanovení všech nových opatření se provádí reevaluace všech hodnotících atributů. Cílem je zjistit, jestli byla provedená opatření natolik účinná, že je možné ohodnotit jednotlivé atributy SOD nižší známkou a tím také snížit prioritu rizika. 
\end{enumerate}

\section{Dokumentace výsledků(7. krok)}
Dokumentace výsledků je posledním krokem v rámci vypracování FMEA. Jeho účelem je sumarizace dosažených výsledků a provedených zlepšení v rámci návrhu produktu nebo výrobního procesu. Jak již bylo řečeno předpokladem pro provedení dokumentace je to, že byly uzavřeny všechny nalezené rizika. Dokumentace výsledků již není součástí vyplňovaného formuláře. Dokument může mít libovolný formát vyhovující potřebám firmy. FMEA může sloužit i jako záruka kvality při komunikaci se zákazníkem v roli obchodního řetězce prodávájící daný produkt. Pro tyto účely je možné také použít toto zhodnocení vytvořené v rámci tohto kroku. Zhodnocení také může sloužit pro informování vysoko postavených osob ve vedení firmy o stavu produktu v rámci kontroly kvality. 
  
%\input{Chapters/Introduction.tex}
%\input{Chapters/SampleChapter1.tex}
%\input{Chapters/SampleChapter2.tex}
%\input{Chapters/TechnicalDetails.tex}
%\input{Chapters/Conclusion.tex}

% Seznam literatury
\printbibliography[title={Liter`atura}, heading=bibintoc]

% Prilohy
\appendix
%\input{Chapters/Appendix1.tex}
%\input{Chapters/Appendix2.tex}

% Priloha vlozena primo do hlavniho LaTeX souboru. Ne vsechny prilohy je nutne mit ve zvlastnich souborech.
%\chapter{Dlouhý zdrojový kód}
%\lstinputlisting[label=src:CppExternal,caption={Dlouhý zdrojový kód v jazyce C++ načtený s externího souboru}]{SourceCodes/ArraySortingAlgorithms.cpp}

\end{document}
